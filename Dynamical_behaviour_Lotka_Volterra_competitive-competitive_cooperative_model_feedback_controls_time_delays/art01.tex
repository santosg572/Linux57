\documentclass[12pt]{article}

\usepackage{amsmath}

\begin{document}

Dynamical behaviour of a Lotka–Volterra competitive-competitive–cooperative 
model with feedback controls and time 
delays - 2019

\hfill

ABSTRACT

The aim of this paper is to investigate the dynamical behaviour of a class of three species Lotka–Volterra 
competitive-competitive-cooperative models with feedback controls and time delays. By developing a new analysis 
technique, we obtain some sufficient conditions that ensure these models have the dynamical property of permanence. We 
also give some sufficient conditions that guarantee the global attractivity of positive solutions for this system by 
constructing a new suitable Lyapunov function. Finally, we give some numerical simulations to illustrate our results 
in this paper.

\hrule

El objetivo de este artículo es investigar el comportamiento dinámico de una clase de tres modelos Lotka-Volterra 
competitivo-competitivo-cooperativo con controles de retroalimentación y retardos temporales. Mediante el desarrollo 
de una nueva técnica de análisis, obtenemos condiciones suficientes que garantizan la propiedad dinámica de 
permanencia de estos modelos. También proporcionamos condiciones suficientes que garantizan la atracción global de 
soluciones positivas para este sistema mediante la construcción de una nueva función de Lyapunov adecuada. Finalmente, 
presentamos algunas simulaciones numéricas para ilustrar nuestros resultados.

\hfill

 1. Introduction

\hfill

The modelling and analysis of the dynamics of biological populations by means of differential equations are of the 
primary concern in population growth problems. A well-known and extensively studied class of models in population 
dynamics is the Lotka–Volterra system which models certain types of interactions among various species. In the real 
world, the growth rate of a natural species will not often respond immediately to changes in its own species or that 
of an interacting species, but will rather do so after a time lag. Time delays are introduced to make the model 
respond better to impersonal law (see, [1–11]).

\hrule

El modelado y análisis de la dinámica de poblaciones biológicas mediante ecuaciones diferenciales es fundamental en 
los problemas de crecimiento poblacional. Un modelo bien conocido y ampliamente estudiado en dinámica poblacional es 
el sistema Lotka-Volterra, que modela ciertos tipos de interacciones entre diversas especies. En el mundo real, la 
tasa de crecimiento de una especie natural no suele responder inmediatamente a los cambios en su propia especie o en 
la de una especie con la que interactúa, sino que lo hará con un cierto desfase temporal. Se introducen desfases 
temporales para que el modelo responda mejor a la ley impersonal (véase [1–11]).

\hrule

Lu et al. in [2] proposed and studied the following Lotka–Volterra system with discrete delays

\hrule

Lu et al. en [2] propusieron y estudiaron el siguiente sistema Lotka-Volterra con retrasos discretos

\[
\left\{\begin{matrix}
\dot{x}_1(t) = x_1(t)[r_1 - a_1 x_1 (t) - a_{11} x_1 (t - \tau_{11} ) + a_{12} x_2 (t - \tau_{12} ) \\
\dot{x}_2(t) = x_2(t)[r_2 - a_2 x_2 (t) - a_{21} x_1 (t - \tau_{21} ) + a_{22} x_2 (t - \tau_{22} )
\end{matrix}\right.
\]

(1)

with initial conditions

\[
x_i(t) = \phi_i(t) \geq 0, t \in [-\tau_0 , 0]; \phi_i(0) > 0, (i = 1, 2)
\]


where $r_i, a_i, a_{ij}$ and $\tau_{ij}$ are constants with $a_i > 0$, $a_{ij} \geq 0$ (i, j = 1, 2) and  0 = max {  ij : i, j = 1, 2 
} , $\phi_{ij}$ is continuous on [-$\tau$ 0 , 0]. They show that delays can change the permanence for Lotka–Volterra cooperative 
systems. For certain delays with the same length, the delayed system has a similar property to the corresponding 
system without delays in the sense of permanence, but for a general delay case, the delays may destroy the permanence 
for the system. In 2010, Nakata and Muroya considered the following nonautonomous Lotka–Volterra cooperative systems 
with time delays (see, [3])

\hrule

donde $r_i, a_i, a_{ij}$ y $\tau_{ij}$ son constantes con $a_i > 0, a_{ij} \geq 0$ (i, j = 1, 2) y $\tau_0$ = máx{$\tau_{ij}$: i, j = 1, 2} , $\phi_{ij}$ es 
continuo en $[-\tau 0 , 0]$. Demuestran que los retrasos pueden cambiar la permanencia de los sistemas cooperativos 
Lotka-Volterra. Para ciertos retrasos con la misma duración, el sistema retrasado tiene una propiedad similar a la del 
sistema correspondiente sin retrasos en cuanto a la permanencia, pero para un caso general de retraso, los retrasos 
pueden destruir la permanencia del sistema. En 2010, Nakata y Muroya consideraron los siguientes sistemas cooperativos 
Lotka-Volterra no autónomos con retrasos temporales (véase [3]).

\[
\left\{\begin{matrix}
\dot{x}(t) = x_1(t)[r_1(t) - a_{11}^1(t)x_1(t) - \tau) - a_{11}^2(t)x_1 (t - 2 \tau) + a_{12}^1(t)x_2 (t - 
\tau)], \\
\dot{x}(t) = x_2(t)[r_2(t) + a_{21}^0(t)x_1(t) + a_21^1(t)x_1 (t - \tau) - a_{22}^0(t)x_2(t) - a_{22}^1(t) x_2 (t - \tau)
\end{matrix}\right.
\]


where $x_i(t)$ (i = 1, 2) denote the density of i-species at time t, $\tau$ is a positive constant and $r_i(t), a_{ij}^l(t) (1 \leq i, j \leq 2; 0 \leq l \leq 2)$ are continuous, bounded and strictly positive functions as $t \in [- \tau, + \infty)$. They 
obtained some sufficient conditions for the permanence of the system (2). Xu and Zu [4] investigated the following 
two-species delayed competitive model with stage structure and harvesting

\hrule

Donde $x_i(t)$ (i = 1, 2) denota la densidad de i-especies en el tiempo t, $\tau$ es una constante positiva y $r_i(t), a_{ij}^l(t) 
(1 \leq i, j \leq 2; 0 \leq l \leq 2)$ son funciones continuas, acotadas y estrictamente positivas cuando $t \in [- \tau, + \infty]$. 
Obtuvieron algunas condiciones suficientes para la permanencia del sistema (2). Xu y Zu [4] investigaron el siguiente 
modelo competitivo retardado de dos especies con estructura de etapas y cosecha.


\[
\left\{\begin{matrix}
\frac{dx_1}{dt} = \alpha(t)x_2(t) - \gamma x_1 (t) - \alpha (t - \tau)e^{\gamma \tau} x_2 (t - \tau), \\
\frac{dx_2(t)}{dt}= \alpha(t - \tau)e^{-\gamma \tau} x_2 (t - \tau) - \beta(t)x_2^2(t) - a_1(t)x_2 
(t)\gamma(t) - E(t)x_2 (t), \\
\frac{dy(t)}{dt}= y(t)(r_1(t) - a_2(t)x_2(t) - b(t)y(t)).
\end{matrix}
\right.
\]


By using the differential inequality theory, some new sufficient conditions which ensure the permanence of the system 
are established. In [5], the authors considered the following competitor-competitor–mutualist Lotka–Volterra systems 
with discrete time delays

\hrule

Mediante la teoría de la desigualdad diferencial, se establecen nuevas condiciones suficientes que garantizan la 
permanencia del sistema. En [5], los autores consideraron los siguientes sistemas Lotka-Volterra 
competidor-competitivo-mutualista con retardos discretos.

\[
\left\{\begin{matrix}
\dot{x}_1(t) = x_1(t)[r_1 (t) - a_{11}^1(t)x_1 (t - \tau) - a_{11}^2(t)x_1 (t - 2 \tau) - a_{12}(t)x_2 (t - 
2\tau) + a_{13}(t)x_3 (t - \tau)], \\\dot{x}_2(t) = x_2(t) [r_2(t) - a_{21}(t)x_1(t - 2 \tau) - a_{22}^1(t)x_2(t - \tau) - a_{22}^2(t)x_2(t - 
2 \tau) + a_{23}(t)x_3 (t - \tau)], \\
\dot{x} = x_3(t)[r_3 (t) + a_{31}(t)x_1 (t - \tau) + a_{32} (t)x 2 (t - \tau) - a_{33}^1(t) x_3 (t) - a_{33}^2(t) x_3 (t - \tau)]. \\
\end{matrix}\right.
\]


(4)

And some sufficient conditions which guarantee the boundedness, permanence and global attraction for system (4) were 
obtained. In 2011, Xu et al. [6] studied the dynamical behaviours for the following Lokta–Volterra predator–prey model 
with two delays

\hrule

Se obtuvieron condiciones suficientes que garantizan la acotación, la permanencia y la atracción global del sistema 
(4). En 2011, Xu et al. [6] estudiaron los comportamientos dinámicos del siguiente modelo depredador-presa de 
Lokta-Volterra con dos retrasos.

\[
\left\{\begin{matrix}
\dot{x}_1(t) = x_1(t)[r_1 - a_{11} x_1 (t - \tau_1 ) - a_{12} y(t - \tau_2 )], 
\dot{x}_2(t) = x_2(t)[-r_2 - a_{21} x_1 (t - \tau_1 ) - a_{22} y(t - \tau_2 )].
\end{matrix}\right.
\]

(5)

Its linear stability and Hopf bifurcation are investigated by analysing the associated characteristic transcendental 
equation. Some explicit formulate for determining the stability and the direction of the Hopf bifurcation periodic 
solutions are obtained by using normal form theory and centre manifold theory.

\hrule

Se investiga su estabilidad lineal y la bifurcación de Hopf mediante el análisis de la ecuación trascendental 
característica asociada. Se obtienen fórmulas explícitas para determinar la estabilidad y la dirección de las 
soluciones periódicas de la bifurcación de Hopf mediante la teoría de la forma normal y la teoría de la variedad 
central.

\hrule

One can find that an ecosystem in the real world is continuously distributed by some forces, which can result in 
changes in the biological parameters such as survival rates. The practical interest in ecology is the question of 
whether or not an ecosystem can withstand those disturbances which persist for a finite period of time. In the control 
systems, we regard the disturbance functions as control variables. These are of significance in the control of ecology 
balance. One of the methods to research it is to alter the system structurally by introducing feedback control 
variables. The feedback control mechanism might be implemented by means of some biological control schemes or 
harvesting procedure. In fact, during the last decade, the qualitative behaviour of the population dynamics with 
feedback control has been studied extensively. In 2009, Nie et al. [12] considered the following non-autonomous 
predator–prey Lotka–Volterra system with feedback controls

\hrule

En la vida real, un ecosistema se distribuye continuamente por ciertas fuerzas, lo que puede provocar cambios en 
parámetros biológicos como las tasas de supervivencia. El interés práctico en ecología radica en determinar si un 
ecosistema puede soportar perturbaciones que persisten durante un período finito. En los sistemas de control, 
consideramos las funciones de perturbación como variables de control. Estas son importantes para controlar el 
equilibrio ecológico. Uno de los métodos para investigarlo es alterar la estructura del sistema mediante la 
introducción de variables de control de retroalimentación. El mecanismo de control de retroalimentación podría 
implementarse mediante esquemas de control biológico o procedimientos de cosecha. De hecho, durante la última década, 
se ha estudiado ampliamente el comportamiento cualitativo de la dinámica poblacional con control de retroalimentación. 
En 2009, Nie et al. [12] consideraron el siguiente sistema Lotka-Volterra depredador-presa no autónomo con controles 
de retroalimentación.


\end{document}

