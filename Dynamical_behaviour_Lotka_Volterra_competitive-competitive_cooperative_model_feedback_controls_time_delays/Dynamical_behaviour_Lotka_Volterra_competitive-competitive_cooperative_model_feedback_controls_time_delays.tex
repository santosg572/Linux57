\documentclass[12pt]{article}

\usepackage{amsmath}
\usepackage{comment}

\begin{document}

Dynamical behaviour of a Lotka–Volterra competitive-competitive–cooperative 
model with feedback controls and time 
delays - 2019

\hfill

ABSTRACT

The aim of this paper is to investigate the dynamical behaviour of a class of three species Lotka–Volterra 
competitive-competitive-cooperative models with feedback controls and time delays. By developing a new analysis 
technique, we obtain some sufficient conditions that ensure these models have the dynamical property of permanence. We 
also give some sufficient conditions that guarantee the global attractivity of positive solutions for this system by 
constructing a new suitable Lyapunov function. Finally, we give some numerical simulations to illustrate our results 
in this paper.

\hrule

El objetivo de este artículo es investigar el comportamiento dinámico de una clase de tres modelos Lotka-Volterra 
competitivo-competitivo-cooperativo con controles de retroalimentación y retardos temporales. Mediante el desarrollo 
de una nueva técnica de análisis, obtenemos condiciones suficientes que garantizan la propiedad dinámica de 
permanencia de estos modelos. También proporcionamos condiciones suficientes que garantizan la atracción global de 
soluciones positivas para este sistema mediante la construcción de una nueva función de Lyapunov adecuada. Finalmente, 
presentamos algunas simulaciones numéricas para ilustrar nuestros resultados.

\hfill

 1. Introduction

\hfill

The modelling and analysis of the dynamics of biological populations by means of differential equations are of the 
primary concern in population growth problems. A well-known and extensively studied class of models in population 
dynamics is the Lotka–Volterra system which models certain types of interactions among various species. In the real 
world, the growth rate of a natural species will not often respond immediately to changes in its own species or that 
of an interacting species, but will rather do so after a time lag. Time delays are introduced to make the model 
respond better to impersonal law (see, [1–11]).

\hrule

El modelado y análisis de la dinámica de poblaciones biológicas mediante ecuaciones diferenciales es fundamental en 
los problemas de crecimiento poblacional. Un modelo bien conocido y ampliamente estudiado en dinámica poblacional es 
el sistema Lotka-Volterra, que modela ciertos tipos de interacciones entre diversas especies. En el mundo real, la 
tasa de crecimiento de una especie natural no suele responder inmediatamente a los cambios en su propia especie o en 
la de una especie con la que interactúa, sino que lo hará con un cierto desfase temporal. Se introducen desfases 
temporales para que el modelo responda mejor a la ley impersonal (véase [1–11]).

\hrule

Lu et al. in [2] proposed and studied the following Lotka–Volterra system with discrete delays

\hrule

Lu et al. en [2] propusieron y estudiaron el siguiente sistema Lotka-Volterra con retrasos discretos

\[
\left\{\begin{matrix}
\dot{x}_1(t) = x_1(t)[r_1 - a_1 x_1 (t) - a_{11} x_1 (t - \tau_{11} ) + a_{12} x_2 (t - \tau_{12} ) \\
\dot{x}_2(t) = x_2(t)[r_2 - a_2 x_2 (t) - a_{21} x_1 (t - \tau_{21} ) + a_{22} x_2 (t - \tau_{22} )
\end{matrix}\right.
\]

(1)

with initial conditions

\[
x_i(t) = \phi_i(t) \geq 0, t \in [-\tau_0 , 0]; \phi_i(0) > 0, (i = 1, 2)
\]


where $r_i, a_i, a_{ij}$ and $\tau_{ij}$ are constants with $a_i > 0$, $a_{ij} \geq 0$ (i, j = 1, 2) and  0 = max {  ij : i, j = 1, 2 
} , $\phi_{ij}$ is continuous on [-$\tau$ 0 , 0]. They show that delays can change the permanence for Lotka–Volterra cooperative 
systems. For certain delays with the same length, the delayed system has a similar property to the corresponding 
system without delays in the sense of permanence, but for a general delay case, the delays may destroy the permanence 
for the system. In 2010, Nakata and Muroya considered the following nonautonomous Lotka–Volterra cooperative systems 
with time delays (see, [3])

\hrule

donde $r_i, a_i, a_{ij}$ y $\tau_{ij}$ son constantes con $a_i > 0, a_{ij} \geq 0$ (i, j = 1, 2) y $\tau_0$ = máx{$\tau_{ij}$: i, j = 1, 2} , $\phi_{ij}$ es 
continuo en $[-\tau 0 , 0]$. Demuestran que los retrasos pueden cambiar la permanencia de los sistemas cooperativos 
Lotka-Volterra. Para ciertos retrasos con la misma duración, el sistema retrasado tiene una propiedad similar a la del 
sistema correspondiente sin retrasos en cuanto a la permanencia, pero para un caso general de retraso, los retrasos 
pueden destruir la permanencia del sistema. En 2010, Nakata y Muroya consideraron los siguientes sistemas cooperativos 
Lotka-Volterra no autónomos con retrasos temporales (véase [3]).

\[
\left\{\begin{matrix}
\dot{x}(t) = x_1(t)[r_1(t) - a_{11}^1(t)x_1(t) - \tau) - a_{11}^2(t)x_1 (t - 2 \tau) + a_{12}^1(t)x_2 (t - 
\tau)], \\
\dot{x}(t) = x_2(t)[r_2(t) + a_{21}^0(t)x_1(t) + a_21^1(t)x_1 (t - \tau) - a_{22}^0(t)x_2(t) - a_{22}^1(t) x_2 (t - \tau)
\end{matrix}\right.
\]


where $x_i(t)$ (i = 1, 2) denote the density of i-species at time t, $\tau$ is a positive constant and $r_i(t), a_{ij}^l(t) (1 \leq i, j \leq 2; 0 \leq l \leq 2)$ are continuous, bounded and strictly positive functions as $t \in [- \tau, + \infty)$. They 
obtained some sufficient conditions for the permanence of the system (2). Xu and Zu [4] investigated the following 
two-species delayed competitive model with stage structure and harvesting

\hrule

Donde $x_i(t)$ (i = 1, 2) denota la densidad de i-especies en el tiempo t, $\tau$ es una constante positiva y $r_i(t), a_{ij}^l(t) 
(1 \leq i, j \leq 2; 0 \leq l \leq 2)$ son funciones continuas, acotadas y estrictamente positivas cuando $t \in [- \tau, + \infty]$. 
Obtuvieron algunas condiciones suficientes para la permanencia del sistema (2). Xu y Zu [4] investigaron el siguiente 
modelo competitivo retardado de dos especies con estructura de etapas y cosecha.


\[
\left\{\begin{matrix}
\frac{dx_1}{dt} = \alpha(t)x_2(t) - \gamma x_1 (t) - \alpha (t - \tau)e^{\gamma \tau} x_2 (t - \tau), \\
\frac{dx_2(t)}{dt}= \alpha(t - \tau)e^{-\gamma \tau} x_2 (t - \tau) - \beta(t)x_2^2(t) - a_1(t)x_2 
(t)\gamma(t) - E(t)x_2 (t), \\
\frac{dy(t)}{dt}= y(t)(r_1(t) - a_2(t)x_2(t) - b(t)y(t)).
\end{matrix}
\right.
\]


By using the differential inequality theory, some new sufficient conditions which ensure the permanence of the system 
are established. In [5], the authors considered the following competitor-competitor–mutualist Lotka–Volterra systems 
with discrete time delays

\hrule

Mediante la teoría de la desigualdad diferencial, se establecen nuevas condiciones suficientes que garantizan la 
permanencia del sistema. En [5], los autores consideraron los siguientes sistemas Lotka-Volterra 
competidor-competitivo-mutualista con retardos discretos.

\[
\left\{\begin{matrix}
\dot{x}_1(t) = x_1(t)[r_1 (t) - a_{11}^1(t)x_1 (t - \tau) - a_{11}^2(t)x_1 (t - 2 \tau) - a_{12}(t)x_2 (t - 
2\tau) + a_{13}(t)x_3 (t - \tau)], \\\dot{x}_2(t) = x_2(t) [r_2(t) - a_{21}(t)x_1(t - 2 \tau) - a_{22}^1(t)x_2(t - \tau) - a_{22}^2(t)x_2(t - 
2 \tau) + a_{23}(t)x_3 (t - \tau)], \\
\dot{x} = x_3(t)[r_3 (t) + a_{31}(t)x_1 (t - \tau) + a_{32} (t)x 2 (t - \tau) - a_{33}^1(t) x_3 (t) - a_{33}^2(t) x_3 (t - \tau)]. \\
\end{matrix}\right.
\]


(4)

And some sufficient conditions which guarantee the boundedness, permanence and global attraction for system (4) were 
obtained. In 2011, Xu et al. [6] studied the dynamical behaviours for the following Lokta–Volterra predator–prey model 
with two delays

\hrule

Se obtuvieron condiciones suficientes que garantizan la acotación, la permanencia y la atracción global del sistema 
(4). En 2011, Xu et al. [6] estudiaron los comportamientos dinámicos del siguiente modelo depredador-presa de 
Lokta-Volterra con dos retrasos.

\[
\left\{\begin{matrix}
\dot{x}_1(t) = x_1(t)[r_1 - a_{11} x_1 (t - \tau_1 ) - a_{12} y(t - \tau_2 )], 
\dot{x}_2(t) = x_2(t)[-r_2 - a_{21} x_1 (t - \tau_1 ) - a_{22} y(t - \tau_2 )].
\end{matrix}\right.
\]

(5)

Its linear stability and Hopf bifurcation are investigated by analysing the associated characteristic transcendental 
equation. Some explicit formulate for determining the stability and the direction of the Hopf bifurcation periodic 
solutions are obtained by using normal form theory and centre manifold theory.

\hrule

Se investiga su estabilidad lineal y la bifurcación de Hopf mediante el análisis de la ecuación trascendental 
característica asociada. Se obtienen fórmulas explícitas para determinar la estabilidad y la dirección de las 
soluciones periódicas de la bifurcación de Hopf mediante la teoría de la forma normal y la teoría de la variedad 
central.

\hrule

One can find that an ecosystem in the real world is continuously distributed by some forces, which can result in 
changes in the biological parameters such as survival rates. The practical interest in ecology is the question of 
whether or not an ecosystem can withstand those disturbances which persist for a finite period of time. In the control 
systems, we regard the disturbance functions as control variables. These are of significance in the control of ecology 
balance. One of the methods to research it is to alter the system structurally by introducing feedback control 
variables. The feedback control mechanism might be implemented by means of some biological control schemes or 
harvesting procedure. In fact, during the last decade, the qualitative behaviour of the population dynamics with 
feedback control has been studied extensively. In 2009, Nie et al. [12] considered the following non-autonomous 
predator–prey Lotka–Volterra system with feedback controls

\hrule

En la vida real, un ecosistema se distribuye continuamente por ciertas fuerzas, lo que puede provocar cambios en 
parámetros biológicos como las tasas de supervivencia. El interés práctico en ecología radica en determinar si un 
ecosistema puede soportar perturbaciones que persisten durante un período finito. En los sistemas de control, 
consideramos las funciones de perturbación como variables de control. Estas son importantes para controlar el 
equilibrio ecológico. Uno de los métodos para investigarlo es alterar la estructura del sistema mediante la 
introducción de variables de control de retroalimentación. El mecanismo de control de retroalimentación podría 
implementarse mediante esquemas de control biológico o procedimientos de cosecha. De hecho, durante la última década, 
se ha estudiado ampliamente el comportamiento cualitativo de la dinámica poblacional con control de retroalimentación. 
En 2009, Nie et al. [12] consideraron el siguiente sistema Lotka-Volterra depredador-presa no autónomo con controles 
de retroalimentación.

\begin{comment}
\[
\left\{\begin{matrix}
\dot{x}_1(t) = x_1(t)[b_1(t) - a_{11}(t) x_1(t) - a_{12}(t) x_2(t) + c_1(t) u_1(t)],
\dot{x}_2(t) = x_2(t)[-b_2(t) + a 21 (t)x 1 (t) − a 22 (t)x 2 (t) − c 2 (t)u 2 (t)], ⎪ ⎪ ⎨ u ˙ 1 (t) = f 1 (t) − e 1 (t)u 1 (t) 
− d 1 (t)x 1 (t), ⎪ ⎩ u ˙ 2 (t) = −e 2 (t)u 2 (t) + d 2 (t)x 2 (t),
\end{matrix}\right.
\]

\end{comment}

(6)

where x1(t) is the prey population density and x2(t) is the predator population density, b1(t) and a11(t) are 
the intrinsic growth rate and density-dependent coefficient of the prey, respectively; b2(t) and a 22 (t) are the 
intrinsic growth rate and densitydependent coefficient of the predator, respectively; a12(t) is the capturing rate 
of the predator and a21(t) is the rate of conversion of nutrient into the reproduction of the predator; ui(t)(i = 
1, 2) are control variables. They studied whether or not the feedback controls have an influence on the permanence of 
a positive solution of the general nonautonomous predator–prey Lotka–Volterra-type systems and establish the general 
criteria on the permanence of system (6), which is independent of some feedback controls. In addition, by constructing 
an appropriate Lyapunov function, some sufficient conditions are obtained for the global stability of any positive 
solution to system (6). In [13], Yang, Wang and Chen proposed and studied the following cooperation system with 
feedback controls

\hrule

donde $x_1(t)$ es la densidad de población de la presa y $x_2(t)$ es la densidad de población del depredador, $b_1(t)$ y $a_{11}(t)$ 
son la tasa de crecimiento intrínseca y el coeficiente dependiente de la densidad de la presa, respectivamente; $b_2(t)$ 
y $a_{22}(t)$ son la tasa de crecimiento intrínseca y el coeficiente dependiente de la densidad del depredador, 
respectivamente; $a_{12}(t)$ es la tasa de captura del depredador y a21(t) es la tasa de conversión de nutrientes en la 
reproducción del depredador; $u_i(t)$ (i =1, 2) son variables de control. Estudiaron si los controles de retroalimentación 
tienen o no influencia en la permanencia de una solución positiva de los sistemas generales no autónomos 
depredador-presa tipo Lotka-Volterra y establecen los criterios generales sobre la permanencia del sistema (6), que es 
independiente de algunos controles de retroalimentación. Además, al construir una función de Lyapunov apropiada, se 
obtienen algunas condiciones suficientes para la estabilidad global de cualquier solución positiva al sistema (6). En 
[13], Yang, Wang y Chen propusieron y estudiaron el siguiente sistema de cooperación con controles de 
retroalimentación

\begin{comment}
x ˙ 1 (t) = x 1 (b 1 − a 11 x 1 (t) + a 12 x 2 (t) − α 1 u 1 (t)), ⎧ ⎪ ⎪ ⎪ x ˙ 2 (t) = x 2 (b 2 + a 21 x 1 (t) 
− a 22 x 2 (t) − α 2 u 2 (t)), ⎪ ⎪ ⎨ u ˙ 1 (t) = −η 1 u 1 (t) + a 1 x 1 (t), ⎪ ⎩ u ˙ 2 (t) = −η 2 u 2 (t) + a 
2 x 2 (t),
\end{comment}

(7)

where $b_i, a_{ij}, \alpha_i, \eta_i, a_i$, i, j = 1, 2 are positive constants. xi(t), (i = 1, 2) are the densities of the 
species at time t, ui(t), (i = 1, 2) denote feedback controls. They showed that if system (7) has a positive 
equilibrium, then feedback controls can only influence the position of the positive equilibrium, and have no influence 
on the stability. In 2018, Wang et al. [14] considered the following three-species Lokta–Volterra predator–prey system 
with feedback

\hrule

Donde $b_i, a_{ij}, \alpha_i, \eta_i, a_i$, i, j = 1, 2 son constantes positivas. $x_i(t)$, (i = 1, 2) son las densidades de las especies 
en el tiempo t, $u_i(t)$, (i = 1, 2) representan controles de retroalimentación. Demostraron que si el sistema (7) tiene 
un equilibrio positivo, entonces los controles de retroalimentación solo pueden influir en la posición del equilibrio 
positivo y no en la estabilidad. En 2018, Wang et al. [14] consideraron el siguiente sistema depredador-presa de tres 
especies de Lokta-Volterra con retroalimentación.

 \begin{comment}
a 12 (t)x 2 (t) ⎪ ⎪ ⎧ ⎪ ⎪ x ˙ 1 (t) = x 1 (t)[r 1 (t) − a 11 (t)x 1 (t) ⎪ ⎪ b 12 (t)x 2 (t) + x 1 (t) ⎪ ⎪ ⎪ 
⎪ a 13 (t)x 3 (t) ⎪ ⎪ − (t)u 1 (t)], ⎪ ⎪ b 13 (t)x 3 (t) + x 1 (t) − d1  ⎪ ⎪ ⎪ ⎪ a 21 (t)x 1 (t) ⎪ ⎪ x ˙ 
2 (t) = x 2 (t)[−r 2 (t) + (t)x 3 (t) + d 2 (t)u 2 (t)], a23  b 12 (t)x 2 (t) + x 1 (t) ⎪ ⎪ ⎨ a 31 (t)x 1 (t) ⎪ 
⎪ x ˙ 3 (t) = x 3 (t)[−r 3 (t) + (t)x 2 (t) + d 3 (t)u 3 (t)], ⎪ ⎪ b 13 (t)x 3 (t) x 1 (t) − a32  + ⎪ ⎪ ⎪ ⎪ 
⎪ ⎪ u ˙ 1 (t) = e 1 (t) − f 1 (t)u 1 (t) + q 1 (t)x 1 (t), ⎪ ⎪ ⎪ ⎪ ⎪ ⎪ u ˙ 2 (t) = e 2 (t) − f 2 (t)u 2 (t) 
− q 2 (t)x 2 (t), ⎪ ⎪ ⎩ u ˙ 3 (t) = e 3 (t) − f 3 (t)u 3 (t) − q 3 (t)x 3 (t),
\end{comment}

(8)

By using a comparison theorem and constructing a suitable Lyapunov function as well as developing some new analysis 
techniques, the authors established a set of easily verifiable sufficient conditions which guarantee the permanence of 
the system and the global attractivity of positive solution for the predator–prey system (8). Furthermore, some 
conditions for the existence, uniqueness and stability of a positive periodic solution for the corresponding periodic 
system were obtained by using the fixed point theory and some new analysis method. More work on feedback controls can 
be found in (cf. [15–21] and the references cited therein). As is known to all, the Lotka–Volterra system with time 
delay and feedback control can respond better to impersonal law. In recent years, more and more attention has been 
paid to some ecosystem models with both feedback control and time delay (see, [22–27]). In 1993, Gopalsamy et al. [22] 
studied a class of autonomous single-species ecosystem with feedback control and time delay

\hrule

Mediante el uso de un teorema de comparación y la construcción de una función de Lyapunov adecuada, así como el 
desarrollo de nuevas técnicas de análisis, los autores establecieron un conjunto de condiciones suficientes fácilmente 
verificables que garantizan la permanencia del sistema y la atracción global de una solución positiva para el sistema 
depredador-presa (8). Además, se obtuvieron algunas condiciones para la existencia, unicidad y estabilidad de una 
solución periódica positiva para el sistema periódico correspondiente mediante el uso de la teoría del punto fijo y un 
nuevo método de análisis. Se puede encontrar más información sobre controles de retroalimentación en (cf. [15–21] y 
las referencias citadas). Como es bien sabido, el sistema Lotka-Volterra con retardo temporal y control de 
retroalimentación puede responder mejor a la ley impersonal. En los últimos años, se ha prestado cada vez más atención 
a algunos modelos de ecosistemas con control de retroalimentación y retardo temporal (véase [22–27]). En 1993, 
Gopalsamy et al. [22] estudiaron una clase de ecosistema autónomo monoespecífico con control de retroalimentación y 
retardo temporal.

\begin{comment}
a1  ⎪ ⎪ ⎧ dn(t) n(t) + a 2 n(t − τ) = rn(t)[1 − ( ) − cu(t)], dt K ⎪ ⎪ ⎨ du(t) = −au(t) + bn(n − τ), ⎩ dt
\end{comment}

(9)

where u(t) denotes an indirect control variable, $\tau, a_2 , a, b, c, r \in (0, \infty)$ and $a_1 \in [0, \infty)$. Some sufficient 
conditions were obtained for the global asymptotic stability of the positive equilibrium for the system (9). In order 
to show that whether the feedback control variables play an essential role on the persistent property of 
Lotka–Volterra cooperative systems or not, Xu and Chen [26] established and studied the following system with time 
delay and feedback control


\begin{comment}

⎪ ⎪ ⎧ x˙ 1 (t) = x 1 (t)[r 1 (t) − a 1 (t)x 1 (t) − a 11 (t)x 1 (t − τ) + a 12 (t)x 2 (t − τ) − b 1 (t)u 1 (t 
− σ 1 )], x˙ 2 (t) = x 2 (t)[r 2 (t) − a 2 (t)x 2 (t) + a 21 (t)x 1 (t) − a 22 (t)x 2 (t − τ) − b 2 (t)u 2 (t − 
σ 2 )], ⎪ ⎪ ⎨u˙ 1 (t) = −c 1 (t)u 1 (t) + d 1 (t)x 1 (t − η 1 ), ⎩ u˙ 2 (t) = −c 2 (t)u 2 (t) + d 2 (t)x 2 (t 
− η 2 ).
\end{comment}

(10)

They obtained some new sufficient conditions which ensured the system to be permanent, and showed that feedback 
control variables had no influence on the permanence of the

system. In 2017, Xu and Li [27] considered the following competition and cooperation model of two enterprises with 
multiple delays and feedback controls

\begin{comment}
dx 1 (t) ⎪ ⎪ ⎧ ⎪ ⎪ dt = x 1 (t)[r 1 (t) − a 1 (t)x 1 (t) − b 1 (t)(x 2 (t) − c 2 (t)) 2 − e 1 (t)u 1 (t − τ 
1 (t))], ⎪ ⎪ ⎪ ⎪ du 1 (t) ⎪ ⎪ = −α 1 (t)u 1 (t) + β 1 (t)x 1 (t − σ 1 (t)), dt ⎪ ⎪ ⎨ dx 2 (t) ⎪ ⎪ = x 2 
(t)[r 2 (t) − a 2 (t)x 2 (t) + b 2 (t)(x 1 (t) − c 1 (t)) 2 − e 2 (t)u 2 (t − τ 2 (t))], ⎪ ⎪ dt ⎪ ⎪ ⎪ ⎪ du 2 
(t) = −α 2 (t)u 2 (t) + β 2 (t)x 2 (t − σ 2 (t)). ⎩ dt
\end{comment}

(11) Some sufficient conditions that guarantee the existence of a unique globally asymptotically stable nonnegative 
almost periodic solution for the system (11) were obtained by constructing a suitable Lyapunov functional and using 
the comparison theorem of differential equations.

However, as far as we know, no work has been done until now for the three-species Lotka–Volterra system with feedback 
control and time delay. Motivated by the above work, we propose and investigate the following three species 
Lotka–Volterra competitive–cooperative model with feedback controls and time delays

\begin{comment}
x ˙ 1 (t) = x 1 (t)[r 1 (t) − a 11 1 (t)x 1 (t − τ) − a 11 2 (t)x 1 (t − 2τ) − a 12 (t)x 2 (t − 2τ) ⎧ ⎪ ⎪ ⎪ 
⎪ a 13 (t)x 3 (t − τ) − d 1 (t)u 1 (t)], + ⎪ ⎪ ⎪ ⎪ x ˙ 2 (t) = x 2 (t)[r 2 (t) − a 21 (t)x 1 (t − 2τ) − a 22 
1 (t)x 2 (t − τ) − a 22 2 (t)x 2 (t − 2τ) ⎪ ⎪ ⎪ ⎪ + a 23 (t)x 3 (t − τ) + d 2 (t)u 2 (t)], x ˙ 3 (t) = x 3 
(t)[r 3 (t) + a 31 (t)x 1 (t − τ) + a 32 (t)x 2 (t − τ) − a 33 1 (t)x 3 (t) ⎪ ⎪ ⎨ ⎪ ⎪ − a 33 2 (t)x 3 (t − 
τ) + d 3 (t)u 3 (t)], ⎪ ⎪ ⎪ ⎪ u ˙ 1 (t) = e 1 (t) − f 1 (t)u 1 (t) + q 1 (t)x 1 (t), ⎪ ⎪ ⎪ ⎪ u ˙ 2 (t) = e 2 
(t) − f 2 (t)u 2 (t) − q 2 (t)x 2 (t), ⎩ u ˙ 3 (t) = e 3 (t) − f 3 (t)u 3 (t) − q 3 (t)x 3 (t),
\end{comment}

(12)

where $x_i(t)$, i = 1, 2, 3 stands for the densities of the species at time t, and $u_i(t)$, i = 1, 2, 3 are the 
indirect control variables. The given coefficients $a_{12}(t), a_{13}(t), a_{21}(t), a_{23}(t), a_{31}(t), a_{32}(t), r_i 
(t), d_i(t), e_i(t), f_i(t), q_i(t)$ (i = 1, 2, 3), $a_{11}^l(t), a_{22}^l(t), a_{33}^l (t)$, (l = 1, 2) are continuous, 
bounded and strictly positive functions on $[0, \infty). r_i(t)$, (i = 1, 2, 3) denote the intrinsic growth rate of the 
i-th species at time t. Especially, $a_{ii}^l(t)$, (i = 1, 2, 3, l = 1, 2) denote the internal competitive coefficient of 
the three species at time t. $a_{12}(t), a_{21}(t)$ are the competitive coefficient of species $x_1(t)$ and $x_2(t)$ at time 
t, respectively. $a_{13}(t), a_23(t), a_{31}(t), a_{32}(t)$ are the cooperative coefficient of species $x_i(t)$, (i = 1, 2, 
3) at time t, respectively. $\tau$ is a positive constant.

Due to the biological interpretation of the system (8), it is reasonable to consider only positive solution of the 
system (8), in other words, take admissible initial conditions

\begin{comment}
x i (t) = φ i (t), i = 1, 2, 3 for t ∈ [−2τ, 0) and φ 1 (0) >0,
\end{comment}

(13)

\begin{comment}
u i (t) = ϕ i (t), i = 1, 2, 3 for t > 0 and ϕ 1 (0) >0.
\end{comment}

Obviously, the solutions of system (12) with the initial values (13) are positive for all $t \geq  0$.

Comparing the systems (4) and (12), one could see that we introduce the control variables $u_i(t)$ (i = 1, 2, 3) so as 
to implement a feedback control mechanism. Our main purpose in this paper is to establish some sufficient conditions 
which ensure the system to be permanence and global attractivity by constructing a new appropriate Lyapunov function 
and developing a new analysis technique. This paper is organized as follows: In Section 2, we provide the conditions 
for the permanence to system (12). In Section 3, by constructing a nonnegative Lyapunov function, we shall derive 
sufficient conditions for the global attractivity of positive solution for the Lotka–Volterra 
Competitive-Competitive–Cooperative model (12). Some numerical simulations to the system are given in Section 4.

2. Permanence

In order to establish a permanence result for the system (12), we need some preparations. Firstly, we introduce the 
following notations and definitions. Given a function g(t) defined on $[t_0 + \infty)$, we set

\[
g^m = sup { g(t) | t_0 < t < + \infty } , g^l = inf { g(t) | t_0 < t < + \infty } .
\]

Definition 2.1: System (12) is called permanent, if there exist positive constants $M_i , N_i , m_i , n_i$ (i = 1, 2, 
3), and T, such that $m_i \leq x_i(t) \leq M_i , n_i \leq u_i(t) \leq N_i$ for any positive solution $Z(t) = (x_1 (t), x_2(t), 
x_3(t), u_1(t), u_2(t), u_3(t))$ of (8) as t > T.

As a direct corollary of Lemma 2.1 of Chen [1], we have.

Lemma 2.1: If a > 0, b > 0 and $\dot{x} \geq b - ax$, when $t \geq 0$ and x(0) > 0, we have

\[
\lim_{t \to + \infty} inf x(t) \geq b/a.
\]

If a > 0, b > 0 and $\dot{x} \geq  b - ax$, when $t \geq 0$ and x(0) > 0, we have

\[
\lim_{t \to + \infty} sup x(t) \geq b/a.
\]


Lemma 2.2 (see [3], Lemma 2.2): Assume that for y(t) > 0, it holds that

\[
\dot{y}(t) \leq y(t)( \lambda - \sum_{i=0}^m \mu^l y(t - l \tau)) + D, 
\]

with initial conditions $y(t) = \phi(t) \geq 0$ for $t \in [- m \tau, 0)$ and $\phi(0) > 0$, where

\[
\lambda > 0, \mu^l \geq 0 \text{(l = 0, 1, 2, · · · , m)}, \mu = \sum_{l=0}^m \mu^l > 0 \text{ and } D \geq 0, 
\]

are constants. Then there exist a positive constant $M_y < + \infty$ such that

\[
\lim_{t \to + \infty} \sup y(t) \geq M_y = - \frac{D}{\lambda} + (\frac{D}{\lambda} + y^*) exp( \lambda m \tau) < + \infty,
\]

(14)

where $y = y^*$ is the unique solution of equation $y( \lambda - \mu y) + D = 0$.

Lemma 2.3 (see [3], Lemma 2.3): Assume that for y(t) > 0, it holds that

\[
\dot{y}(t) \geq y(t)[ \lambda - \sum_{i=0}^m \mu^l y(t - l \tau)]. 
\]

If the system (14) holds, then, there exists a positive constant $m_y > 0$ such that for $\mu = \sum_{i=0}^m \mu^l > 0$


\[
\lim_{t \to + \infty} inf y(t) \geq m_y = \frac{\lambda}{\mu}  \exp ( \lambda - \mu M-y) m \tau) > 0
\]

For the system (12), let

\begin{comment}
m m e e N 2 = 2 ,N 3 = 3 , l l f 2 f 3

(r1 m  + r 3 m + d3 m N3 )2  P 1 = exp { (r 1 m + r 3 m + d 3 m N 3 )τ } , a11 1l a33 2l 

(r2 m  + r 3 m + d2 m N 2 + d3 m N3 )2  P 2 = exp { (r 2 m + r 3 m + d 2 m N 2 + d 3 m N 3 )τ } , a22 1l a33 2l 

a13 m  P 1 P 1 + q m M1  a13 m  e1 m  1 M 1 = − + ( + x 1 ∗ ) exp(2r 1 m τ), N 1 = , l r 1 m r 1 m f 1

a P a P M 2 = − + ( + x 2 ∗ ) exp(2(r 2 m + d 2 m N 2 )τ), r 2 m + m 23 dm 2 2 N 2 r 2 m + m 23 dm 2 2 N2 

M 3 = x 3 ∗ exp((r 3 m + (a 31 m + a 32 m + d 3 m ) max { M 1 , M 2 , N 3 } )τ),

r1 l  −a12 m M2 −d1 m N1  m 1 = exp[(r 1 l − a 12 m M 2 − d 1 m N 1 − (a 11 1m + a 11 2m )M 1 )2τ], a 11 1m + a11 
2m 

r2 l  −a21 m M1  m 2 = exp[(r 2 l − a 21 m M 1 − (a 22 1m + a 22 2m )M 2 )2τ], a 22 1m + a22 2m 

r3 l  + a31 l m 1 + a32 l m2  m 3 = exp[(r 3 l + a 31 l m 1 + a 32 l m 2 − (a 33 1m + a 33 2m )M 3 )τ], a 33 1m + a33 
2m 

+ q 1 l − 2 − q m m 1 q m M 2 3 M3  e3 l  e1 l  e2 l  n 1 = ,n 2 = ,n 3 = , f 1 m f 2 m f 3 m

where x 1 ∗ is the unique positive solution of equation x 1 [r 1 m − (a 11 1l + a 11 2l )x 1 ] + a 13 m P 1 = 0, x 2 
∗ is the unique positive solution of equation x 2 [r 2 m + d 2 m N 2 − (a 22 1l + a 22 2l )x 2 ] + a 23 m P 2 = 0, 
and x 3 ∗ is the unique positive solution of equation x 3 (t)[r 3 m + (a 31 m + a 32 m + d 3 m ) max { M 1 , M 2 , N 
3 } − (a 33 1l + a 33 2l )x 3 ] = 0.
\end{comment}

Theorem 2.1: Assume the following conditions satisfy

\begin{comment}
(H 1 ) a 11 2l > a 31 m , (H 2 ) a 12 l > a 32 m , (H 3 ) a 33 1l > a 13 m , (H 4 ) a 21 l > a31 m 

(H 5 ) a 22 2l > a 32 m , (H 6 ) a 33 1l > a 23 m , (H 7 ) r 1 l > a 12 m M 2 + d 1 m N 1 ,

(H 8 ) r 2 l > a 21 m M 1 , (H 9 ) e 2 l > q 2 m M 2 , (H 10 ) e 3 l > q 3 m M 3 .
\end{comment}

Then the system (12) is permanent.

Proof. By the fifth equation of system (8), we have

\begin{comment}
u ˙ 2 (t) ≤ e 2 (t) − f 2 (t)u 2 (t) ≤ e 2 m − f 2 l u 2 (t).

e2 m  lim sup u 2 (t) ≤ = N 2 . t→ + ∞ f2 l 
\end{comment}

(15)

Moreover, similar to the above discussion of the fifth equation of system (12), from the sixth of system (12), we have

\[
\lim_{t \to + \infty} \sup u_3(t) \leq \frac{e_3^m}{f_3^l} = N_3
\]

(16)

Next, suppose that $\lim_{t \to + \infty} x_1(t) x_3 (t - \tau) = + \infty$, then there exists a time sequence
${t_k}_{k=1}^{\infty}$ such that

\[
\lim_{k \to + \infty} \sup x_1 (t_k ) x_3 (t_k - \tau) = + \infty,
\]

(17)

and

\[
\frac{d}{dt} (x_1(t) x_3 (t - \tau )) |_{t=t_k} \geq  0, k = 1, 2, · · · . 
\]


(18)

From system (12), one has

\begin{comment}
d (x1  (t)x 3 (t − τ)) = x 1 (t)x 3 (t − τ)[r 1 (t) + r 3 (t − τ) − a 11 1 (t)x 1 (t − τ) dt

− a 11 2 (t)x 1 (t − 2τ) − a 12 (t)x 2 (t − 2τ) + a 13 (t)x 3 (t − τ)

− d 1 (t)u 1 (t) + a 31 (t − τ)x 1 (t − 2τ) + a 32 (t − τ)x 2 (t − 2τ)

− a 33 1 (t − τ)x 3 (t − τ) − a 33 2 (t−τ)x 3 (t−2τ) + d 3 (t−τ)u 3 (t−τ)]

≤ x 1 (t)x 3 (t − τ)[r 1 m + r 3 m + d 3 m N 3 − (a 11 2l − a 31 m )x 1 (t − 2τ)

− (a 12 l − a 32 m )x 2 (t − 2τ) − (a 33 1l − a 13 m )x 3 (t − τ)

(19)


− a 11 1l x 1 (t − τ) − a 33 2l x 3 (t − 2τ) − d 1 l u 1 (t)].

\end{comment}

From (18), (19), we can obtain

\begin{comment}
(a 11 2l − a 31 m )x 1 (t k − 2τ) + (a 12 l − a 32 m )x 2 (t k − 2τ) + (a 33 1l − a 13 m )x 3 (t k − τ) + d 1 l 
u 1 (t k ) + a 11 1l x 1 (t k − τ) + a 33 2l x 3 (t k − 2τ) ≤ r 1 m + r 3 m + d 3 m N 3 .
\end{comment}

(20)

Thus, by the assumption of the Theorem 2.1 and (20), it holds that

\begin{comment}
r1 m  + r 3 m + d3 m N 3 + r 3 m + d3 m N3  r1 m  x 1 (t k − τ) ≤ , x 3 (t k − 2τ) ≤ . a 11 1l a33 2l 
\end{comment}

Moreover, by (19) and the assumption of the Theorem 2.1, it follows that

\begin{comment}
d (x1  (t)x 3 (t − τ)) ≤ x 1 (t)x 3 (t − τ)[r 1 m + r 3 m + d 3 m N 3 ]. dt
\end{comment}

(21)

By integrating both sides of (21) from $t_k - \tau$ to $t_k$ further, we have

\begin{comment}
x 1 (t k )x 3 (t k − τ) ≤ x 1 (t k − τ)x 3 (t k − 2τ) exp { (r 1 m + r 3 m + d 3 m N 3 )τ } .
\end{comment}

Therefore

\begin{comment}
(r1 m  + r 3 m + d3 m N3 )2  x 1 (t k )x 3 (t k − τ) ≤ exp { (r 1 m + r 3 m + d 3 m N 3 )τ } . a11 1l a33 2l 
\end{comment}

However, it leads to a contradiction with (17). Thus, we have

\begin{comment}
(r1 m  + r 3 m + d3 m N3 )2  lim sup(x 1 (t)x 3 (t − τ)) ≤ P 1 = exp { (r 1 m + r 3 m + d 3 m N 3 )τ } . (22) t→ + 
∞ a 11 1l a33 2l 
\end{comment}


Moreover, similar to the above discussion, we can also obtain that

\begin{comment}

lim sup(x 2 (t)x 3 (t − τ)) ≤ P2 

t→ + ∞

(r2 m  + r 3 m + d2 m N 2 + d3 m N3 )2  = exp { (r 2 m + r 3 m + d 2 m N 2 + d 3 m N 3 )τ } . a22 1l a33 2l 
\end{comment}


(23)

According to the first equation of system (21) and (22), it follows that

\begin{comment}

x ˙ 1 (t) ≤ x 1 (t)[r 1 − a 11 1 x 1 (t − τ) − a 11 2 x 1 (t − 2τ) + a 13 x 3 (t − τ)]

≤ x 1 (t)[r 1 m − a 11 1l x 1 (t − τ) − a 11 2l x 1 (t − 2τ)] + a 13 m P 1 .
\end{comment}


From Lemma 2.2, we have

\begin{comment}

a13 m  P 1 P1  a13 m  lim sup x 1 (t) ≤ − + ( + x 1 ∗ ) exp(2r 1 m τ) = M 1 , t→ + ∞ r 1 m r1 m 
\end{comment}


(24)

where $x_1^*$  is the unique positive solution of equation $x_1 [r_1^m - (a_{11}^{1l} + a_{11}^{2l} )x_1] + a_{13}^m P_1 = 0$.

Similar to the above discussion of the first equation of system (12), from (23) and the second equation of system 
(12), we obtain

\begin{comment}

x ˙ 2 (t) ≤ x 2 (t)[r 2 − a 22 1 x 2 (t − τ) − a 22 2 x 2 (t − 2τ) + a 23 x 3 (t − τ) + d 2 u 2 (t)]

≤ x 2 (t)[r 2 m − a 22 1l x 2 (t − τ) − a 22 2l x 2 (t − 2τ) + d 2 m N 2 ] + a 23 m P 2 .
\end{comment}


So, we have

\begin{comment}
a P a P lim sup x 2 (t) ≤ − + ( + x 2 ∗ ) exp((r 2 m + d 2 m N 2 )2τ) = M 2 . (25) t→ + ∞ r 2 m + m 23 dm 2 2 N 2 
r 2 m + m 23 dm 2 2 N2 
\end{comment}


where $x_2^*$  is the unique positive solution of the following equation

\begin{comment}

x 2 [r 2 m + d 2 m N 2 − (a 22 1l + a 22 2l )x 2 ] + a 23 m P 2 = 0.
\end{comment}


From the third equation of system (12), we obtain

\begin{comment}

x ˙ 3 (t) = x 3 (t)[r 3 + a 31 x 1 (t − τ) + a 32 x 2 (t − τ) − a 33 1 x 3 (t) − a 33 2 x 3 (t − τ) + d 3 u 3 
(t)]

≤ x 3 (t)[r 3 m + a 31 m M 1 + a 32 m M 2 + d 3 m N 3 − a 33 1l x 3 (t) − a 33 2l x 3 (t − τ)]

≤ x 3 (t)[r 3 m + (a 31 m + a 32 m + d 3 m ) max { M 1 , M 2 , N 3 } − a 33 1l x 3 (t) − a 33 2l x 3 (t − τ)].
\end{comment}


By Lemma 2.2, it holds that

\begin{comment}

lim sup x 3 (t) ≤ x 3 ∗ exp((r 3 m + (a 31 m + a 32 m + d 3 m ) max { M 1 , M 2 , N 3 } )τ) = M 3 .

(26)

t→ + ∞
\end{comment}


where $x_3^*$  is the unique positive solution of the following equation

\begin{comment}
x 3 (t)[r 3 m + (a 31 m + a 32 m + d 3 m ) max { M 1 , M 2 , N 3 } − (a 33 1l + a 33 2l )x 3 ] = 0.
\end{comment}

From the fourth equation of system (12), one has

\begin{comment}
u ˙ 1 (t) ≤ e 1 m − f 1 l u 1 (t) + q 1 m M 1 .
\end{comment}

Thus, according to Lemma 2.1, it follows that

\begin{comment}
e 1 m + q1 m M1  lim sup u 1 (t) ≤ = N 1 . t→ + ∞ f1 l 
\end{comment}

(27)

On the contrary, from the first equation of system (12), we have

\begin{comment}
x ˙ 1 (t) ≥ x 1 (t)[r 1 − a 11 1 x 1 (t − τ) − a 11 2 x 1 (t − 2τ) − a 12 x 2 (t − 2τ) − d 1 (t)u 1 (t)]

≥ x 1 (t)[r 1 l − a 11 1m x 1 (t − τ) − a 11 2m x 1 (t − 2τ) − a 12 m M 2 − d 1 m N 1 ].
\end{comment}

By Lemma 2.3, one easily verifies that

\begin{comment}
lim inf x 1 (t) t→ + ∞

r1 l −a12 m M2 −d1 m N1  ≥ exp[2(r 1 l − a 12 m M 2 − d 1 m N 1 − (a 11 1m + a 11 2m )M 1 )τ] = m 1 .
\end{comment}

By the same way, from the second and third equations of system (12), we deduce

\begin{comment}
x ˙ 2 (t) ≥ x 2 (t)[r 2 − a 21 x 1 (t − 2τ) − a 22 1 x 2 (t − τ) − a 22 2 x 2 (t − 2τ)]

≥ x 2 (t)[r 2 l − a 21 m M 1 − a 22 1m x 2 (t − τ) − a 22 2m x 2 (t − 2τ)],

x ˙ 3 (t) ≥ x 3 (t)[r 3 + a 31 x 1 (t − τ) + a 32 x 2 (t − τ) − a 33 1 x 3 (t) − a 33 2 x 3 (t − τ)]

≥ x 3 (t)[r 3 l + a 31 l m 1 + a 32 l m 2 − a 33 1m x 3 (t) − a 33 2m x 3 (t − τ)].
\end{comment}

Thus, by Lemma 2.3, we have

\begin{comment}
r2 l  −a21 m M1  lim inf x 2 (t) ≥ exp[(r 2 l − a 21 m M 1 − (a 22 1m + a 22 2m )M 2 )2τ] = m 2 , t→ + ∞ a 22 1m 
+ a22 2m 
\end{comment}

(29)

and

\begin{comment}
lim inf x 3 (t) t→ + ∞

r 3 l + a31 l m 1 + a32 l m2  ≥ exp[(r 3 l + a 31 l m 1 + a 32 l m 2 − (a 33 1m + a 33 2m )M 3 )τ] = m 3 . (30) a 33 
1m + a33 2m 
\end{comment}

According to the fourth equation of system (12), we have

\[
\dot{u}_1 (t) \geq e_1^l - f_1^m u_1(t) + q_1^l m_1 .
\]

From Lemma 2.1, we can obtain

\[
\lim_{t \to + \infty} \inf u_1 (t) \geq \frac{e_1^l + q_1^l m_1}{f_1^m}  = n_1 .
\]


(31)

Similarly, from the fifth and sixth equations of system (12), it follows that

\begin{comment}
u ˙ 2 (t) ≥ e 2 l − f 2 m u 2 (t) − q 2 m M 2 , u ˙ 3 (t) ≥ e 3 l − f 3 m u 3 (t) − q 3 m M 3 .
\end{comment}

Moreover, by Lemma 2.1, it follows that

\begin{comment}
e2 l  −q2 m M2  lim inf u 2 (t) ≥ = n 2 , t→ + ∞ f2 m 
\end{comment}

(32)

and

\begin{comment}
e3 l  −q3 m M3  lim inf u 3 (t) ≥ = n 3 . t→ + ∞ f3 m 
\end{comment}

(33)

From (15), (16), and (24)–(33), this completes the proof of Theorem 2.1.

3. Globally attractive

In this section, we shall prove that the system (12) is globally attractive. To get the sufficient conditions for 
globally attractive of system (12), we give firstly the following definition and Lemma.

Definition 3.1: System (12) is said to be globally attractive, if there exists a positive solution $X(t) = (x_1(t), x_ 
2 (t), x_3(t), u_1 (t), u_2 (t), u_3 (t))$ of the system (12) such that

\[
\lim_{t \to + \infty} | x_i (t) - y_i (t) | = 0, \lim_{t \to + \infty} | u_i (t) - v_i (t) | = 0,
\]


for any other positive solution $Y(t) = (Y_1 (t), Y_2 (t), Y_3 (t), v_1 (t), v_2 (t), v_3 (t))$ of the system (12).

Lemma 3.1 (See [28], Lemma 8.2): If the function $f(t) : R^+ \to R$ is uniformly continuous, and the limit $\lim_{t \to + \infty} \int_0^t f(s)ds$ 
exists and is finite, then $\lim_{t \to + \infty} f (t) = 0$.


Theorem 3.1: Let $a_{11}^M = max { a_{11}^{1m}, a_{11}^{2m} } , a_{22}^M = \max { a_{22}^{1m} , a_{22}^{2m} }$ . Assume that the system (12) 
satisfies $(H_1 ) - (H_{10})$ and the following conditions satisfy

\[
(H_{11} ) \lim_{t \to + \infty} \inf A_i(t) > 0, B_i > 0, \text{(i = 1, 2, 3)}
\]


where

\begin{comment}
2 t A 1 (t) = a 11 1l + a 11 2l − ∑ a 11 k (s + kτ)ds[r 1 m + (a 11 1m + a 11 2m )M1  ∫ k=1 t−kτ

2 t + kτ + a 12 m M 2 + a 13 m M 3 + d 1 m N 1 ] − M 1 ∑ a 11 k (s + kτ)ds × a 11 k (t + kτ) ∫ k=1 ( t )

2 − 2τM 1 a 11 M ∑ a 11 k (t + kτ) − (1 + τM 2 (a 22 1m + 2a 22 2m ))a 21 (t + 2τ) k=1

− (1 + τM 3 a 33 2m )a 31 (t + τ) − q 1 m ,

2 t A 2 (t) = a 22 1l + a 22 2l − ∑ a 22 k (s + kτ)ds[r 2 m + a 21 m M 1 + (a 22 1m + a 22 2m )M2  ∫ k=1 t−kτ

2 t + kτ + a 23 m M 3 + d 2 m N 2 ] − M 2 ∑ a 22 k (s + kτ)ds × a 22 k (t + kτ) ∫ k=1 ( t )

2 − 2τM 2 a 22 M ∑ a 22 k (t + kτ) k=1

− (1 + τM 1 (a 11 1m + 2a 11 2m ))a 12 (t + 2τ) − (1 + τM 3 a 33 2m )a 32 (t + τ) − q 2 m ,

t A 3 (t) = a 33 1l + a 33 2l − (τM 3 a 33 2m )(a 33 1m + a 33 2 (t + τ)) − a 33 2 (s + τ)ds[r 3 m + a 31 m M1  ∫ 
t−τ

+ a 32 m M 2 + (a 33 1m + a 33 2m )M 3 + d 3 m N 3 ] − (1 + τM 1 (a 11 1m + 2a 11 2m ))a 13 (t + τ)

− (1 + τM 2 (a 22 1m + 2a 22 2m ))a 23 (t + τ) − q 3 m ,

B 1 = f 1 l − (1 + τM 1 (a 11 1m + 2a 11 2m ))d 1 m ,

B 2 = f 2 l − (1 + τM 2 (a 22 1m + 2a 22 2m ))d 2 m ,

B 3 = f 3 l − (1 + τM 3 a 33 2m )d 3 m .
\end{comment}

Then system (12) is globally attractive.

Proof. Suppose that (x 1 (t), x 2 (t), x 3 (t), u 1 (t), u 2 (t), u 3 (t)) and (y 1 (t), y 2 (t), y 3 (t), v 1 (t), v 
2 (t), v 3 (t)) are any two different positive solutions of the system (12). Then from Theorem 2.1, there exist 
positive constants M i , N i , m i , n i , i = 1, 2, 3 and T, such that

\begin{comment}
m i ≤ x i (t), y i (t) ≤ M i , n i ≤ u i (t), v i (t) ≤ N i , i = 1, 2, 3, for all t ≥ T.
\end{comment}

Define

\begin{comment}
V 11 (t) = | ln x 1 (t) − ln y 1 (t) | .
\end{comment}

Calculating the upper right derivative of V 11 (t) along the solution of system (12), we obtain

\begin{comment}
D + V 11 (t) = sgn(x 1 (t) − y 1 (t))[−a 11 1 (t)(x 1 (t − τ) − y 1 (t − τ))

− a 11 2 (t)(x 1 (t − 2τ) − y 1 (t − 2τ))

− a 12 (t)(x 2 (t − 2τ) − y 2 (t − 2τ)) + a 13 (t)(x 3 (t − τ) − y 3 (t − τ))

− d 1 (t)(u 1 (t) − v 1 (t))]

= sgn(x 1 (t) − y 1 (t))[−(a 11 1 (t) + a 11 2 (t))(x 1 (t) − y 1 (t)) − d 1 (t)(u 1 (t) − v 1 (t))

− a 12 (t)(x 2 (t − 2τ) − y 2 (t − 2τ)) + a 13 (t)(x 3 (t − τ) − y 3 (t − τ))

t t ˙1  + a 11 1 (t) (˙x 1 (θ) − y ˙ 1 (θ)dθ + a 11 2 (t) (˙x 1 (θ) − y (θ)dθ] ∫ t−τ ∫ t−2τ

= sgn(x 1 (t) − y 1 (t))[−(a 11 1 (t) + a 11 2 (t))(x 1 (t) − y 1 (t)) − d 1 (t)(u 1 (t) − v 1 (t))

− a 12 (t)(x 2 (t − 2τ) − y 2 (t − 2τ)) + a 13 (t)(x 3 (t − τ) − y 3 (t − τ))

2 t + ∑ a 11 k (t) ∫ t−kτ (x 1 (θ)[r 1 (θ) − a 11 1 (θ)x 1 (θ − τ) k=1

− a 11 2 (θ)x 1 (θ − 2τ) − a 12 (θ)x 2 (θ − 2τ)

+ a 13 (θ)x 3 (θ − τ) − d 1 (θ)u 1 (θ)] − y 1 (θ)[r 1 (θ) − a 11 1 (θ)y 1 (θ − τ)

− a 11 2 (θ)y 1 (θ − 2τ)

−a 12 (θ)y 2 (θ − 2τ) + a 13 (θ)y 3 (θ − τ) − d 1 (θ)v 1 (θ)])dθ]

= sgn(x 1 (t) − y 1 (t))[−(a 11 1 (t) + a 11 2 (t))(x 1 (t) − y 1 (t)) − d 1 (t)(u 1 (t) − v 1 (t))

− a 12 (t)(x 2 (t − 2τ) − y 2 (t − 2τ)) + a 13 (t)(x 3 (t − τ) − y 3 (t − τ))

2 t + ∑ a 11 k (t) ∫ t−kτ ((x 1 (θ) − y 1 (θ))[r 1 (θ) − a 11 1 (θ)y 1 (θ − τ)

−a 11 2 (θ)y 1 (θ − 2τ)

− a 12 (θ)y 2 (θ − 2τ) + a 13 (θ)y 3 (θ − τ) − d 1 (θ)v 1 (θ)]

+ x 1 (θ)[−a 11 1 (θ)(x 1 (θ − τ)

− y 1 (θ − τ)) − a 11 2 (θ)(x 1 (θ − 2τ) − y 1 (θ − 2τ)) − a 12 (θ)(x 2 (θ − 2τ)

− y 2 (θ − 2τ))

+ a 13 (θ)(x 3 (θ − τ) − y 3 (θ − τ)) − d 1 (θ)(u 1 (θ) − v 1 (θ))dθ]

≤ −(a 11 1 (t) + a 11 2 (t)) | x 1 (t) − y 1 (t) | + d 1 (t) | u 1 (t)

− v 1 (t) | + a 12 (t) | x 2 (t − 2τ) − y 2 (t − 2τ) |

+ a 13 (t) | x 3 (t − τ) − y 3 (t − τ) |

2 t + ∑ (a 11 k (t) ∫ t−kτ ([r 1 (θ) + a 11 1 (θ)y 1 (θ − τ) + a 11 2 (θ)y 1 (θ − 2τ) k=1

+ a 12 (θ)y 2 (θ − 2τ) + a 13 (θ)y 3 (θ − τ) + d 1 (θ)v 1 (θ)] | x 1 (θ) − y 1 (θ) |

+ x 1 (θ)[a 11 1 (θ)

× | x 1 (θ − τ) − y 1 (θ − τ) | + a 11 2 (θ) | x 1 (θ − 2τ) − y 1 (θ − 2τ) |

+ a 12 (θ) | x 2 (θ − 2τ) − y 2 (θ − 2τ) |
\end{comment}

(34)

\begin{comment}
+ a 13 (θ) | x 3 (θ − τ) − y 3 (θ − τ) | + d 1 (θ) | u 1 (θ) − v 1 (θ) | ])dθ).

Next, we define that 2 t t V 12 (t) = ∑ a 11 k (s + kτ)([r 1 (θ) + a 11 1 (θ)y 1 (θ − τ) + a 11 2 (θ)y 1 (θ − 2τ) 
∫ ∫ k=1 t−kτ s

+ a 12 (θ)y 2 (θ − 2τ) + a 13 (θ)y 3 (θ − τ) + d 1 (θ)v 1 (θ)] | x 1 (θ) − y 1 (θ) |

+ x 1 (θ)[a 11 1 (θ) | x 1 (θ − τ) − y 1 (θ − τ) | + a 11 2 (θ) | x 1 (θ − 2τ) − y 1 (θ − 2τ) |

+ a 12 (θ) | x 2 (θ − 2τ) − y 2 (θ − 2τ) | + a 13 (θ) | x 3 (θ − τ) − y 3 (θ − τ) |

(35)

+ d 1 (θ) | u 1 (θ) − v 1 (θ) | )dθds.
\end{comment}

Then, from (34) and (35), we have

\begin{comment}
2 ∑ V 1i (t) ≤ −(a 11 1 (t) + a 11 2 (t)) | x 1 (t) − y 1 (t) | + d 1 (t) | u 1 (t) − v 1 (t) | i=1

+ a 12 (t) | x 2 (t − 2τ) − y 2 (t − 2τ) | + a 13 (t) | x 3 (t − τ) − y 3 (t − τ) |

2 t + ∑ ∫ t−kτ a 11 k (s + kτ)ds[r 1 (t) + (a 11 1 (t) + a 11 2 (t))M 1 + a 12 (t)M2  k=1

2 t + a 13 (t)M 3 + d 1 (t)N 1 ] | x 1 (t) − y 1 (t) | + M 1 ∑ a 11 k (s + kτ)ds ∫

× [a 11 1 (t) | x 1 (t − τ) − y 1 (t − τ) | + a 11 2 (t) | x 1 (t − 2τ) − y 1 (t − 2τ) |

+ a 12 (t) | x 2 (t − 2τ) − y 2 (t − 2τ) | + a 13 (t) | x 3 (t − τ) − y 3 (t − τ) |

+ d 1 (t) | u 1 (t) − v 1 (t) | ]

≤ −(a 11 1 (t) + a 11 2 (t)) | x 1 (t) − y 1 (t) | + (1 + M 1 τ(a 11 1m + 2a 11 2m ))d 1 (t) | u 1 (t) − v 1 (t) |

+ a 12 (t) | x 2 (t − 2τ) − y 2 (t − 2τ) | + a 13 (t) | x 3 (t − τ) − y 3 (t − τ) |

2 t + ∑ ∫ t−kτ a 11 k (s + kτ)ds[r 1 (t) + (a 11 1 (t) + a 11 2 (t))M 1 + a 12 (t)M2  k=1

2 t + a 13 (t)M 3 + d 1 (t)N 1 ] | x 1 (t) − y 1 (t) | + M 1 ∑ ( a 11 k (s + kτ)ds ∫ k=1 t−kτ 2 × a 11 k (t) | x 1 
(t − kτ) − y 1 (t − kτ) | ) + 2τM 1 a 11 M ∑ (a 11 k (t) | x 1 (t − kτ) k=1

−y 1 (t − kτ) | )

+ M 1 τ(a 11 1m + 2a 11 2m )[a 12 (t) | x 2 (t − 2τ) − y 2 (t − 2τ) | + a 13 (t) | x 3 (t − τ)

(36)

−y 3 (t − τ) | ].
\end{comment}


Define

\begin{comment}
2 t w + kτ V 13 (t) = M 1 ∑ a 11 k (s + kτ)a 11 k (w + kτ) | x 1 (w) − y 1 (w) | dsdw ∫ ∫ k=1 t−kτ w

2 t + 2τM 1 a 11 M ∑ a 11 k (w + kτ) | (x 1 (w) − y 1 (w) | dw ∫ k=1 t−kτ

t + (1 + τM 1 (a 11 1m + 2a 11 2m )) a 12 (w + 2τ) | x 2 (w) − y 2 (w) | dw ∫ t−2τ

t + (1 + τM 1 (a 11 1m + 2a 11 2m )) a 13 (w + τ) | x 3 (w) − y 3 (w) | dw. ∫ t−τ
\end{comment}

(37)

Let

V 1 (t) = V 11 (t) + V 12 (t) + V 13 (t).

(38)

According to (36) and (37), calculating the upper right derivative of V 1 (t), we have

\begin{comment}
2 t D + V 1 (t) ≤ − { a 11 1l + a 11 2l − ∑ a 11 k (s + kτ)ds[r 1 m + (a 11 1m + a 11 2m )M1  ∫ k=1 t−kτ

2 t + kτ + a 12 m M 2 + a 13 m M 3 + d 1 m N 1 ] − M 1 ∑ ( a 11 k (s + kτ)ds × a 11 k (t + kτ)) ∫ k=1 t

2 − 2τM 1 a 11 M ∑ a 11 k (t + kτ) }| x 1 (t) − y 1 (t) | + (1 + τM 1 (a 11 1m + 2a 11 2m ))a 12 (t + 2τ) k=1

× | x 2 (t) − y 2 (t) | + (1 + τM 1 (a 11 1m + 2a 11 2m ))a 13 (t + τ) | x 3 (t) − y 3 (t) |

(39)

+ (1 + M 1 τ(a 11 1m + 2a 11 2m ))d 1 m | u 1 (t) − v 1 (t) | .

\end{comment}

Similarly, we define $V_{21}(t) = | \ln x_2 (t) - \ln y_2(t) |$ , then one obtain

\begin{comment}
D + V 21 (t) = sgn(x 2 (t) − y 2 (t))[−a 21 (t)(x 1 (t − 2τ) − y 1 (t − 2τ)) − a 22 1 (t)(x 2 (t − τ) − y 2 (t 
− τ))

− a 22 2 (t)(x 2 (t − 2τ) − y 2 (t − 2τ)) + a 23 (t)(x 3 (t − τ) − y 3 (t − τ)) + d 2 (t)(u 2 (t) − v 2 (t)]

= sgn(x 2 (t) − y 2 (t))[−(a 22 1 (t) + a 22 2 (t))(x 2 (t) − y 2 (t)) − a 21 (t)(x 1 (t − 2τ) − y 1 (t − 2τ))

t ˙2  + a 23 (t)(x 3 (t − τ) − y 3 (t − τ)) + d 2 (t)(u 2 (t) − v 2 (t)) + a 22 1 (t) (˙x 2 (θ) − y (θ))dθ ∫ 
t−τ

t ˙2  + a 22 2 (t) (˙x 2 (θ) − y (θ))dθ] ∫ t−2τ

= sgn(x 2 (t) − y 2 (t))[−(a 22 1 (t) + a 22 2 (t))(x 2 (t) − y 2 (t)) − a 21 (t)(x 1 (t − 2τ) − y 1 (t − 2τ))

2 t + a 23 (t)(x 3 (t − τ) − y 3 (t − τ)) + d 2 (t)(u 2 (t) − v 2 (t)) + ∑ a 22 k (t) (x 2 (θ)[r 2 (θ) ∫ k=1 
t−kτ

− a 21 (θ)x 1 (θ − 2τ) − a 22 1 (θ)x 2 (θ − τ) − a 22 2 (θ)x 2 (θ − 2τ) + a 23 (θ)x 3 (θ − τ) + d 2 (θ)u 2 (θ)]

− y 2 (θ)[r 2 (θ) − a 21 (θ)y 1 (θ − 2τ) − a 22 1 (θ)y 2 (θ − τ) − a 22 2 (θ)y 2 (θ − 2τ) + a 23 (θ)y 3 (θ − 
τ)

+

d 2 (θ)v 2 (θ)])dθ]

= sgn(x 2 (t) − y 2 (t))[−(a 22 1 (t) + a 22 2 (t))(x 2 (t) − y 2 (t)) − a 21 (t)(x 1 (t − 2τ) − y 1 (t − 2τ))

2 t + a 23 (t)(x 3 (t − τ) − y 3 (t − τ)) + d 2 (t)(u 2 (t) − v 2 (t)) + ∑ a 22 k (t) ((x 2 (θ) − y 2 (θ)) ∫ 
k=1 t−kτ

× [r 2 (θ) − a 21 (θ)y 1 (θ − 2τ) − a 22 1 (θ)y 2 (θ − τ) − a 22 2 (θ)y 2 (θ − 2τ) + a 23 (θ)y 3 (θ − τ)

+ d 2 (θ)v 2 (θ)] + x 2 (θ)[−a 21 (θ)(x 1 (θ − 2τ) − y 1 (θ − 2τ)) − a 22 1 (θ)(x 2 (θ − τ) − y 2 (θ − τ))

− a 22 2 (θ)(x 2 (θ − 2τ) − y 2 (θ − 2τ)) + a 23 (θ)(x 3 (θ −τ)−y 3 (θ −τ)) + d 2 (θ)(u 2 (θ)−v 2 (θ))])dθ]

≤ −(a 22 1 (t) + a 22 2 (t)) | x 2 (t) − y 2 (t) | + d 2 (t) | u 2 (t) − v 2 (t) | + a 21 (t) | x 1 (t − 2τ) − y 
1 (t − 2τ) |

2 t + a 23 (t) | x 3 (t − τ) − y 3 (t − τ) | + ∑ (a 22 k (t) ([r 2 (θ) + a 21 (θ)y 1 (θ − 2τ) + a 22 1 (θ)y 2 (θ 
−τ) ∫ k=1 t−kτ

+ a 22 2 (θ)y 2 (θ − 2τ) + a 23 (θ)y 3 (θ − τ) + d 2 (θ)v 2 (θ)] | x 2 (θ) − y 2 (θ) |

+ x 2 (θ)[a 21 (θ) | x 1 (θ − 2τ) − y 1 (θ − 2τ) | + a 22 1 (θ) | x 2 (θ − τ) − y 2 (θ − τ) |

+ a 22 2 (θ) | x 2 (θ − 2τ) − y 2 (θ − 2τ) | + a 23 (θ) | x 3 (θ −τ) − y 3 (θ − τ) | + d 2 (θ) | u 2 (θ)−v 2 
(θ) | ])dθ).
\end{comment}

(40)

On the other hand, define

\begin{comment}
2 t t V 22 (t) = ∑ a 22 k (s + kτ)([r 2 (θ) + a 21 (θ)y 1 (θ − 2τ) + a 22 1 (θ)y 2 (θ − τ) ∫ ∫ k=1 t−kτ s

+ a 22 2 (θ)y 2 (θ − 2τ) + a 23 (θ)y 3 (θ − τ) + d 2 (θ)v 2 (θ)] | x 2 (θ) − y 2 (θ) |

+ x 2 (θ)[a 21 (θ) | x 1 (θ − 2τ) − y 1 (θ − 2τ) | + a 22 1 (θ) | x 2 (θ − τ) − y 2 (θ − τ) |

+ a 22 2 (θ) | x 2 (θ − 2τ) − y 2 (θ − 2τ) | + a 23 (θ) | x 3 (θ − τ) − y 3 (θ − τ) |

(41)

+ d 2 (θ) | u 2 (θ) − v 2 (θ) | ])dθds.

\end{comment}

From (40), (41), we have

\begin{comment}
2 ∑ D + V 2i ≤ −(a 22 1 (t) + a 22 2 (t)) | x 2 (t) − y 2 (t) | + d 2 (t) | u 2 (t) − v 2 (t) | i=1

+ a 21 (t) | x 1 (t − 2τ) − y 1 (t − 2τ) |

2 t + a 23 (t) | x 3 (t − τ) − y 3 (t − τ) | + ∑ a 22 k (s + kτ)ds[r 2 (t) + a 21 (t)M1  ∫ k=1 t−kτ

+ (a 22 1 (t) + a 22 2 (t))M 2 + a 23 (t)M 3 + d 2 (t)N 2 ] | x 2 (t) − y 2 (t) |

2 t + M 2 ∑ a 22 k (s + kτ)ds[a 21 (t) | x 1 (t − 2τ) − y 1 (t − 2τ) | ∫ k=1 t−kτ

+ a 22 1 (t) | x 2 (t − τ) − y 2 (t − τ) |

+ a 22 2 (t) | x 2 (t − 2τ) − y 2 (t − 2τ) | + a 23 (t) | x 3 (t − τ) − y 3 (t − τ) |

+ d 2 (t) | u 2 (t) − v 2 (t) | ]

≤ −(a 22 1 (t) + a 22 2 (t)) | x 2 (t) − y 2 (t) |

+ (1 + τM 2 (a 22 1m + 2a 22 2m ))d 2 (t) | u 2 (t) − v 2 (t) |

+ a 21 (t) | x 1 (t − 2τ) − y 1 (t − 2τ) | + a 23 (t) | x 3 (t − τ) − y 3 (t − τ) |

2 t + ∑ ∫ t−kτ a 22 k (s + kτ)ds[r 2 (t) + a 21 (t)M 1 + (a 22 1 (t) + a 22 2 (t))M2  k=1

2 t + a 23 (t)M 3 + d 2 (t)N 2 ] | x 2 (t) − y 2 (t) | + M 2 ∑ ( a 22 k (s + kτ)ds ∫ k=1 t−kτ 2 × a 22 k (t) | x 2 
(t − kτ) − y 2 (t − kτ) | ) + 2τM 2 a 22 M ∑ (a 22 k (t) | x 2 (t − kτ) k=1

−y 2 (t − kτ) | )

+ τM 2 (a 22 1m + 2a 22 2m )[a 21 (t) | x 1 (t − 2τ) − y 1 (t − 2τ) | + a 23 (t) | x 3 (t − τ)

(42)

−y 3 (t − τ) | ].

\end{comment}

Furthermore, define

\begin{comment}
2 t w + kτ V 23 (t) = M 2 ∑ a 22 k (s + kτ)a 22 k (w + kτ) | x 2 (w) − y 2 (w) | dsdw ∫ ∫ k=1 t−kτ w

+ 2τM 2 a 22 M ∑ a 22 k (w + kτ) | x 2 (w) − y 2 (w) | dw ∫ k=1 t−kτ

t + (1 + τM 2 (a 22 1m + 2a 22 2m )) a 21 (w + 2τ) | x 1 (w) − y 1 (w) | dw ∫ t−2τ

t + (1 + τM 2 (a 22 1m + 2a 22 2m )) a 23 (w + τ) | x 3 (w) − y 3 (w) | dw. ∫ t−τ
\end{comment}

(43)

Let

V 2 (t) = V 21 (t) + V 22 (t) + V 23 (t).

(44)

From (42) and (43), we can get the upper right derivative of V 2 (t)

\begin{comment}
2 t D + V 2 (t) ≤ − { a 22 1l + a 22 2l − ∑ a 22 k (s + kτ)ds[r 2 m + a 21 m M 1 + (a 22 1m + a 22 2m )M2  ∫ k=1 
t−kτ

2 t + kτ + a 23 m M 3 + d 2 m N 2 ] − M 2 ∑ ( a 22 k (s + kτ)ds × a 22 k (t + kτ)) ∫ k=1 t

2 − 2τM 2 a 22 M ∑ a 22 k (t + kτ) } k=1

× | x 2 (t) − y 2 (t) | + (1 + τM 2 (a 22 1m + 2a 22 2m )a 21 (t + 2τ) | x 1 (t) − y 1 (t) |

+ (1 + τM 2 (a 22 1m + 2a 22 2m )a 23 (t + τ) | x 3 (t) − y 3 (t) |

(45)

+ (1 + τM 2 (a 22 1m + 2a 22 2m ))d 2 m | u 2 (t) − v 2 (t) | .

\end{comment}

Similarly, we define

\[
V_{31}(t) = | \ln x_3 (t) - \ln y_3 (t) | .
\]

Then, it follows that

\begin{comment}
D + V 31 = sgn(x 3 (t) − y 3 (t))[a 31 (t)(x 1 (t − τ) − y 1 (t − τ)) + a 32 (t)(x 2 (t − τ)

− y 2 (t − τ)) − a 33 1 (t)(x 3 (t) − y 3 (t)) − a 33 2 (t)(x 3 (t − τ) − y 3 (t − τ))

+ d 3 (t)(u 3 (t) − v 3 (t)]

= sgn(x 3 (t) − y 3 (t))[a 31 (t)(x 1 (t − τ) − y 1 (t − τ)) + a 32 (t)(x 2 (t − τ)

− y 2 (t − τ)) − (a 33 1 (t) + a 33 2 (t))(x 3 (t) − y 3 (t)) + a 33 2 (t)

t × (˙x 3 (θ) − y ˙ 3 (θ))dθ + d 3 (t)(u 3 (t) − v 3 (t))] ∫ t−τ

= sgn(x 3 (t) − y 3 (t))[−(a 33 1 (t) + a 33 2 (t))(x 3 (t) − y 3 (t)) + d 3 (t)(u 3 (t) − v 3 (t))

+ a 31 (t)(x 1 (t − τ) − y 1 (t − τ)) + a 32 (t)(x 2 (t − τ) − y 2 (t − τ))

t + a 33 2 (t) x 3 (θ)[r 3 (θ) + a 31 (θ)x 1 (θ − τ) + a 32 (θ)x 2 (θ − τ) ∫ t−τ

− a 33 1 (θ)x 3 (θ) − a 33 2 (θ)x 3 (θ − τ) + d 3 (θ)u 3 (θ)]

− y 3 (θ)[r 3 (θ) + a 31 (θ)y 1 (θ − τ) + a 32 (θ)y 2 (θ − τ)

− a 33 1 (θ)y 3 (θ) − a 33 2 (θ)y 3 (θ − τ) + d 3 (θ)v 3 (θ)]dθ

= sgn(x 3 (t) − y 3 (t))[−(a 33 1 (t) + a 33 2 (t))(x 3 (t) − y 3 (t)) + d 3 (t)(u 3 (t) − v 3 (t))

+ a 31 (t)(x 1 (t − τ) − y 1 (t − τ)) + a 32 (t)(x 2 (t − τ) − y 2 (t − τ)) + a 33 2 (t)

t × ((x 3 (θ) − y 3 (θ))[r 3 (θ) + a 31 (θ)y 1 (θ − τ) + a 32 (θ)y 2 (θ − τ) ∫ t−τ

− a 33 1 (θ)y 3 (θ) − a 33 2 (θ)y 3 (θ − τ) + d 3 (θ)v 3 (θ)] + x 3 (θ)[a 31 (θ)(x 1 (θ − τ)

− y 1 (θ − τ)) + a 32 (θ)(x 2 (θ − τ) − y 2 (θ − τ)) − a 33 1 (θ)(x 3 (θ)

− y 3 (θ)) − a 33 2 (θ)(x 3 (θ − τ) − y 3 (θ − τ)) + d 3 (θ)(u 3 (θ) − v 3 (θ))])dθ]

≤ −(a 33 1 (t) + a 33 2 (t)) | x 3 (t) − y 3 (t) | + d 3 (t) | u 3 (t) − v 3 (t) | + a 31 (t) | x 1 (t − τ)

− y 1 (t − τ) | + a 32 (t) | (x 2 (t − τ) − y 2 (t − τ) | + a 33 2 (t)

t × ([r 3 (θ) + a 31 (θ)y 1 (θ − τ) + a 32 (θ)y 2 (θ − τ) + a 33 1 (θ)y 3 (θ) ∫ t−τ

+ a 33 2 (θ)y 3 (θ − τ) + d 3 (θ)v 3 (θ)] | x 3 (θ) − y 3 (θ) | + x 3 (θ)[a 31 (θ) | x 1 (θ − τ)

− y 1 (θ − τ) | + a 32 (θ) | x 2 (θ − τ) − y 2 (θ − τ) | + a 33 1 (θ) | x 3 (θ) − y 3 (θ) |

(46)

+ a 33 2 (θ) | x 3 (θ − τ) − y 3 (θ − τ) | + d 3 (θ) | u 3 (θ) − v 3 (θ) | ])dθ.

\end{comment}


Let

\begin{comment}
t t V 32 (t) = a 33 2 (s + τ)([r 3 (θ) + a 31 (θ)y 1 (θ − τ) + a 32 (θ)y 2 (θ − τ) ∫ t−τ ∫ s

+ a 33 1 (θ)y 3 (θ) + a 33 2 (θ)y 3 (θ − τ) + d 3 (θ)v 3 (θ)] | x 3 (θ) − y 3 (θ) |

+ x 3 (θ)[a 31 (θ) | x 1 (θ − τ) − y 1 (θ − τ) | + a 32 (θ) | x 2 (θ − τ) − y 2 (θ − τ) |

+ a 33 1 (θ) | x 3 (θ) − y 3 (θ) | + a 33 2 (θ) | x 3 (θ − τ) − y 3 (θ − τ) |

(47)

+ d 3 (θ) | u 3 (θ) − v 3 (θ) | ])dθds.

\end{comment}

Then, we have

\begin{comment}
2 ∑ D + V 3i ≤ −(a 33 1 (t) + a 33 2 (t)) | x 3 (t) − y 3 (t) | + d 3 (t) | u 3 (t) − v 3 (t) | i=1

+ a 31 (t) | x 1 (t − τ) − y 1 (t − τ) | + a 32 (t) | (x 2 (t − τ) − y 2 (t − τ) |

t + a 33 2 (s + τ)ds[r 3 (t) + a 31 (t)M 1 + a 32 (t)M2  ∫ t−τ

+ (a 33 1 (t) + a 33 2 (t))M 3 + d 3 (t)N 3 ] | x 3 (t) − y 3 (t) |

t + M 3 a 33 2 (s + τ)ds[a 31 (t) | x 1 (t − τ) − y 1 (t − τ) | ∫ t−τ

+ a 32 (t) | x 2 (t − τ) − y 2 (t − τ) | + a 33 1 (t) | x 3 (t) − y 3 (t) |

+ a 33 2 (t) | x 3 (t − τ) − y 3 (t − τ) | + d 3 (t) | u 3 (t) − v 3 (t) | ]

≤ −(a 33 1 (t) + a 33 2 (t) − τM 3 a 33 2m a 33 1 (t)) | x 3 (t) − y 3 (t) |

+ (1 + τM 3 a 33 2m )d 3 (t) | u 3 (t) − v 3 (t) | + a 31 (t) | x 1 (t − τ) − y 1 (t − τ) |

t + a 32 (t) | (x 2 (t − τ) − y 2 (t − τ) | + a 33 2 (s + τ)ds[r 3 (t) ∫ t−τ

+ a 31 (t)M 1 + a 32 (t)M 2 + (a 33 1 (t) + a 33 2 (t))M 3 + d 3 (t)N 3 ] | x 3 (t) − y 3 (t) |

+ τM 3 a 33 2m [a 31 (t) | x 1 (t − τ) − y 1 (t − τ) | + a 32 (t) | x 2 (t − τ) − y 2 (t − τ) |

(48)

+ a 33 2 (t) | x 3 (t − τ)−y 3 (t − τ) | ].

\end{comment}


Take

\begin{comment}
t V 33 (t) = (1 + τM 3 a 33 2m ) a 31 (w + τ) | x 1 (w) − y 1 (w) | dw ∫ t−τ

t + (1 + τM 3 a 33 2m ) a 32 (w + τ) | x 2 (w) − y 2 (w) | dw ∫ t−τ

t + τM 3 a 33 2m a 33 2 (w + τ) | x 3 (w) − y 3 (w) | dw. ∫ t−τ
\end{comment}

(49)

Moreover, we take

V 3 (t) = V 31 (t) + V 32 (t) + V 33 (t).

(50)

Then, we have

\begin{comment}

t D + V 3 (t) ≤ − { a 33 1l + a 33 2l − τM 3 a 33 2m [a 33 1m + a 33 2 (t + τ)] − a 33 2 (s + τ)ds ∫ t−τ

× [r 3 m + a 31 m M 1 + a 32 m M 2 + (a 33 1m + a 33 2m )M 3 + d 3 m N 3 ] }

× | x 3 (t) − y 3 (t) | + (1 + τM 3 a 33 2m )a 31 (t + τ) | x 1 (t) − y 1 (t) |

+ (1 + τM 3 a 33 2m )a 32 (t + τ) | (x 2 (t) − y 2 (t) | + (1 + τM 3 a 33 2m )d 3 m | u 3 (t)−v 3 (t) | .
\end{comment}

(51)

\begin{comment}
V4 

(t)

=

|

ln

u1 

(t)

−

ln

v1 

(t)

|

,

V5 

(t)

=

|

ln

u2 

(t)

−

ln

v2 

(t)

|

,

V6 

(t)

=

|

ln

u3 

(t)−

\end{comment}

Take  v 3 (t) | , and calculate the upper right derivative of V 4 (t), V 5 (t), V 6 (t), we have

\begin{comment}
D + V 4 (t) ≤ sgn(u 1 (t) − v 1 (t))[−f 1 (t)(u 1 (t) − v 1 (t)) + q 1 (t)(x 1 (t) − y 1 (t))
\end{comment}


(52)

\begin{comment}
≤ −f 1 l | u 1 (t) − v 1 (t) | + q 1 m | x 1 (t) − y 1 (t) | ,

D + V 5 (t) ≤ sgn(u 2 (t) − v 2 (t))[−f 2 (t)(u 2 (t) − v 2 (t)) + q 2 (t)(x 2 (t) − y 2 (t))
\end{comment}

(53)

\begin{comment}
≤ −f 2 l | u 2 (t) − v 2 (t) | + q 2 m | x 2 (t) − y 2 (t) | ,

D + V 6 (t) ≤ sgn(u 3 (t) − v 3 (t))[−f 3 (t)(u 3 (t) − v 3 (t)) + q 3 (t)(x 3 (t) − y 3 (t))
\end{comment}

(54)

\begin{comment}
≤ −f 3 l | u 3 (t) − v 3 (t) | + q 3 m | x 3 (t) − y 3 (t) | .
\end{comment}

Moreover, we give a Lyapunov function as follows

\[
V(t) = \sum_{i=1}^6  V_i(t),
\]

from (39), (45), (51)–(54), we can obtain

\[
D^+ V(t) \leq - \sum_{i=1}^3 (A_i (t) | x_i(t) - y_i(t) | + B_i | u_i(t) - v_i (t) | ) 
\]

(55)

for all $t \geq t + \tau$.

In view of the conditions of Theorem 3.1, there exist a constant $\alpha > 0$ and $T^* \geq T + \tau$  such that for all $t \geq T^*$  , it holds 
that

\[
A_i(t) \geq \alpha > 0, B_i \alpha > 0, (i = 1, 2, 3).
\]

(56)

Integrating from $T^*$  to t on both sides of (55) and by (56), we have

\[
V(t) + \alpha \int_{T^*}^t ( \sum_{i=1}^3 [ | x_i(s) - y_i(s) | + | u_i (s) - v_i (s) | ])ds \leq V(T^*) < + \infty .
\]


(57)

Therefore, V(t) is bounded on $[T^* , + \infty)$, and we have

\[
\int_{T^*}^{\infty} ( \sum_{i=1}^3 [ | x_i(t) - y_i(t) | + | u_i (t) - v_i (t) | ])ds \leq \frac{V(T)}{\alpha} < + \infty
\]

(58)

By (58), we have

\[
\sum_{i=1}^3 ( | x_i (t) - y_i (t) | + | u_i (t) - v_i (t) | ) \in  L^1 (T, + \infty).
\]

(59)


From the uniformity permanence of the system (12), $\sum_{i=1}^3 [ | x_i (t) - y_i (t) | + | u_i(t)-v_i (t) | ]$ is uniformity continuous on $[T^*, + \infty)$. By Lemma 3.1, we can obtain

\[
lim_{t \to + \infty} | x_i (t) - y_i(t) | = 0, \lim_{t \to + \infty} | u_i (t) - v_i (t) | = 0, (i = 1, 2, 3). 
\]

This completes the proof of Theorem 3.1.

4. Numerical simulation

In this section, we give some numerical simulations to support our theoretical analysis. As an example, we consider 
the following Lotka–Volterra competitivecompetitive–cooperative model with feedback controls and time delays and 
choose the

Figure 1. The numerical solution of systems (60) with the initial conditions (61) and $\tau = 0.075$.

periodic function as the coefficients of the model

\begin{comment}
x˙ 1 (t) = x 1 (t)[ 2 + | sin(t) | − 5 + | sin(t) | x 1 (t − τ) − 1 + | cos(t) | x 1 (t − 2τ) − 2 + sin(t) 550 x 
2 (t − 2τ) ⎪ ⎪ ⎧ ⎪ ⎪ 2 3 5 + 2 + cos(t) 300 x 3 (t − τ) − (0.015 + 0.001 cos πt)u 1 (t)], ⎪ ⎪ ⎪ ⎪ ⎪ ⎪ x˙ 
2 (t) = x 2 (t)[ 2 + | 2 cos(t) | − 2 + cos(t) 550 x 1 (t − 2τ) − 6 + | 4 cos(t) | x 2 (t − τ) − 1 + | 5 sin(t) | 
x 2 (t − 2τ) ⎪ ⎪ ⎪ + 2 + cos(t) 300 x 3 (t − τ) + (0.0075 + 0.0005 sin πt)u 2 (t)], x˙ 3 (t) = x 3 (t)[ 1 + | 
sin(t) | + 1.5 + cos(t) 1500 x 1 (t − τ) + 1.5 + sin(t) 1500 x 2 (t − τ) − 5 + cos(t) x 3 (t) ⎪ ⎪ ⎨ 2 2 ⎪ ⎪ − 
6 5 + sin(t) | | x 3 (t − τ) + (0.016 + 0.004 sin πt)u 3 (t)], ⎪ ⎪ ⎪ ⎪ u˙ 1 (t) = (0.35 + 0.05 sin πt) − (0.6 + 
0.1 cos πt)u 1 (t) + (0.0016 + 0.0003 sin πt)x 1 (t), ⎪ ⎪ ⎪ ⎪ u˙ 2 (t) = (0.3 + 0.05 cos πt) − (0.7 + 0.2 cos 
πt)u 2 (t) − (0.0015 + 0.0005 sin πt)x 2 (t), ⎪ ⎩ u˙ 3 (t) = (3 + 0.4 sin πt) − (5 + 0.5 sin πt)u 3 (t) − (0.0005 
+ 0.00015 sin πt)x 3 (t).
\end{comment}

(60)

By calculating, we have

\begin{comment}
P 1 ≈ 5.50, P 2 ≈ 6.14, x 1 ∗ ≈ 1.22, x 2 ∗ ≈ 1.09, x 3 ∗ ≈ 0.37,

M 1 ≈ 1.38, M 2 ≈ 1.54, M 3 ≈ 0.4,

N 1 ≈ 0.81, N 2 ≈ 0.7, N 3 ≈ 0.76, m 1 ≈ 0.43, m 2 ≈ 0.51, m 3 ≈ 0.21,

n 1 ≈ 0.52, n 2 ≈ 0.36,

n 3 ≈ 0.47, r 1 l − a 12 m M 2 − d 1 m N 1 ≈ 0.98, r 2 l − a21 m 

M 1 ≈ 0.99, e 2 l − q 2 m M 2 ≈ 0.25, e 3 l − q 3 m M 3 ≈ 2.60,

lim inf A 1 ≈ 0.35,(t) lim inf A 2 (t) ≈ 0.40, lim inf A 3 (t) ≈ 0.48, t→ + ∞ t→ + ∞ t→ + ∞

B 1 ≈ 0.48, B 2 ≈ 0.49, B 3 ≈ 4.48.
\end{comment}

It is easy to show that the system (60) satisfies the conditions of Theorem 3.1. It follows from Theorem 3.1 that the 
Lotka–Volterra competitive-competitive–cooperative model (60) is permanent and globally attractive. By employing the 
software package MATLAB 7.1, we can solve the numerical solutions of the system (60) which are shown in Figures 1–3. 
Figure 1 shows that the permanence of the systems (60) with time delay $\tau = 0.075$ and the

initial conditions

x 1 (t 0 ) = 0.7, x 2 (t 0 ) = 0.7, x 3 (t 0 ) = 0.7, u 1 (t 0 ) = 0.5, u 2 (t 0 ) = 0.5, u 3 (t 0 ) = 0.5. (61)

From Figure 2, it is not difficult to find that the system (60) is globally attractive. Figure 3 shows the dynamical 
behaviour of the systems (60).

5. Conclusion

This paper presents the use of Lyapunov stability theorem for system of nonlinear differential equations. This method 
is a powerful tool for solving nonlinear differential equations in mathematical physics, chemistry and engineering 
etc. The technique constructing an appropriate Lyapunov function provides a new efficient method to handle the 
nonlinear structure with time delay and feedback control.

We have dealt with the problem of positive solution for a class of three-species Lotka–Volterra competitive- 
competitive–cooperative with feedback controls and time delays. By developing some new analysis techniques and 
constructing a new suitable Lyapunov function, we obtain some sufficient conditions which ensure the system to be 
permanent and globally attractive. Our results show that feedback control variables and time delay terms have 
influence both the persistent property and global attractive of system (12). Moreover, some numerical simulations to 
the system (60) are given to illustrate our results obtained in this paper. In particular, the sufficient conditions 
that we obtained are very simple and practical, which provide flexibility for the application and analysis of the 
Lotka–Volterra models with feedback controls and time delays.

Remark: The main contribution and innovation of this paper are as follows: (1) The control variables are introduced to 
the known model (4) to implement a feedback control mechanism, and the new model can better describe the interactions 
among multi-species. Obviously, system (4) is the special case of the new system (12). To the best of the author’s

knowledge, this is the first time such a system is proposed. (2) To study the new model, we obtain some new methods 
and skills (such as the new structure method of the Lyapunov function, the applications of delay differential 
inequalities) that can also be used to research other related models with multi-delays and feedback controls. Because 
of the complexity of the new system, the structure of the Lyapunov function is completed step by step to overcome the 
difficulties brought about by the multiple time delays and feedback controls, please see pages 10–17. (3) In this 
paper, the research contents are richer than the related references. We study not only the permanence and global 
attractivity of the new system but also some numerical simulations to the new system (12) are given to illustrate our 
results obtained in this paper. (4) The sufficient conditions obtained herein are new, general, and easily verifiable, 
which provide flexibility for the application and analysis of three-species multi-delays Lotka–Volterra predator–prey 
model with feedback controls.

Disclosure statement

No potential conflict of interest was reported by the authors.

Funding

This work was supported by Science Fund for Distinguished Young Scholars of China: [Grant Number cstc2014jcyjjq40004]; 
National Nature Science Fund of China: [Grant Number 61503053]; Sichuan Science and Technology Program of China: 
[Grant Number 2018JY0480]; Natural Science Foundation Project of CQ CSTC of China: [Grant Number cstc2015jcyj BX0135].

References

[1] F.D. Chen, On a nonlinear nonautonomous predator-prey model with diffusion and distributed delay, J. Comput. Appl. 
Math. 180 (1) (2005), pp. 33–49.

[2] G.C. Lu, Z.Y. Lu, and X.Z. Lian, Delay effect on the permanence for Lotka-Volterra cooperative systems, Nonlinear 
Anal. Real World Appl. 11 (4) (2010), pp. 2810–2816.

[3] Y. Nakata and Y. Muroya, Permanence for nonautonomous Lotka-Volterra cooperative systems with delays, Nonlinear 
Anal. Real World Appl. 11 (1) (2010), pp. 528–534.

[4] C.J. Xu and Y.S. Zu, Permanence of a two species delayed competitive model with stage structure and harvesting, 
Bull. Korean Math. Soc. 52 (4) (2015), pp. 1069–1076.

[5] G.C. Lu and Z.Y. Lu, Permanence for two-species Lotka-Volterra cooperative systems with delays, Math. Biosci. Eng. 
5 (3) (2008), pp. 477–484.

[6] C.J. Xu, X.H. Tang, M.X. Liao, and X.F. He, Bifurcation analysis in a delayed Lokta-Volterra predator-prey model 
with two delays, Nonlinear Dyn. 66 (1-2) (2011), pp. 169–183.

[7] A. Muhammadhaji, Z.D. Teng, and M. Rehim, Dynamical behavior for a class of delayed competitive- mutualism 
systems, Diff. Equ. Dyn. Sys. 23 (3) (2015), pp. 281–301.

[8] C.J. Xu, X.H. Tang, and M.X. Liao, Stability and bifurcation analysis of a delayed predator-prey model of prey 
dispersal in two-patch environments, Appl. Math. Comput. 216 (10) (2010), pp. 2920–2936.

[9] C.J. Xu and M.X. Liao, Bifurcation behaviours in a delayed three-species food-chain model with Holling type-II 
functional response, Appl. Anal. 92 (12) (2013), pp. 2468–2486.

[10] C.J. Xu and M.X. Liao, Bifurcation analysis of an autonomous epidemic predator-prey model with delay, Annali di 
Matematica Pura ed Applicata 193 (1) (2014), pp. 23–28.

[11] C.J. Xu and P.L. Li, Oscillations for a delayed predator-prey model with Hassell-Varley-type functional response, 
C. R. Biol. 338 (4) (2015), pp. 227–240.

[12] L. Nie, Z. Teng, L. Hu, and J. Peng, Permanence and stability in non-autonomous predator-prey Lotka-Volterra 
systems with feedback controls, Comput. Math. Appl. 58 (2009), pp. 436–448.

[13] K. Yang, H.N. Wang, and F.D. Chen, On the stability property of a Lotka-Volterra cooperation system with feedback 
controls, Mathematica Applicata 27 (2) (2014), pp. 243–247. (in Chinese)

[14] C.Y. Wang, Y.Q. Zhou, Y.H. Li, and R. Li, Well-posedness of a ratio-dependent Lotka-Volterra system with feedback 
control, Bound. Value Probl. 2018 (2018), p. ID: 117.

[15] Y. Fan and L. Wang, Global asymptotical stability of a Logistic model with feedback control, Nonlinear Anal. Real 
World Appl. 11 (2010), pp. 2686–2697.

[16] K. Gopalsamy and P. Weng, Global attractivity in a competition system with feedback controls, Comput. Math. Appl. 
45 (2003), pp. 665–676.

[17] F. Chen, The permanence and global attractivity of Lotka–Volterra competition system with feedback controls, 
Nonlinear Anal. Real World Appl. 7 (2006), pp. 133–143.

[18] J. Li, A. Zhao, and J. Yan, The permanence and global attractivity of a Kolmogorov system with feedback controls, 
Nonlinear Anal Real World Appl. 10 (2009), pp. 506–518.

[19] C.Y. Wang, X.W. Li, and H. Yuan, The permanence of a ratio-dependent Lotka-Volterra predatorprey model with 
feedback control, Adv. Mat. Res. 765–767 (2013), pp. 327–330.

[20] R.Y. Han, F.D. Chen, and X.D. Xie, Stability of Lotka-Volterra cooperation system with single feedback control, 
Ann. Appl. Math. 31 (3) (2015), pp. 287–296.

[21] C.Y. Wang, H. Liu, and S. Pan, Globally attractive of a ratio-dependent Lotka-Volterra predatorprey model with 
feedback control, Adv. Biosci. Bioeng. 4 (5) (2016), pp. 59–66.

[22] K. Gopalsamy and P.X. Weng, Feedback regulation of logistic growth, Int. J. Math. Math. Sci. 16

(1) (1993), pp. 177–192.

[23] F.D. Chen, Z. Li, and Y.J. Huang, Note on the permanence of a competitive system with infinite delay and feedback 
controls, Nonlinear Anal. Real World Appl. 8 (2) (2007), pp. 680–687.

[24] F.D. Chen, J.H. Yang, and L.J. Chen, Note on the persistent property of a feedback control system with delays, 
Nonlinear Anal. Real World Appl. 11 (2) (2010), pp. 1061–1066.

[25] Z. Li, M.A. Han, and F.D. Chen, Influence of feedback controls on an autonomous Lotka-Volterra competitive system 
with infinite delays, Nonlinear Anal. Real World Appl. 14 (1) (2013), pp. 402–413.

[26] J.Y. Xu and F.D. Chen, Permanence of a Lotka-Volterra cooperative system with time delays and feedback controls, 
Commun. Math. Biol. Neurosci. 2015 (2015), p. Article ID: 18.

[27] C.J. Xu and P.L. Li, Almost periodic solutions for a competition and cooperation model of two enterprises with 
time-varying delays and feedback controls, J. Appl. Math. Comput. 53 (1–2) (2017), p. 397411.

[28] H.K. Khalil, Nonlinear Systems, 3rd ed., Prentice-Hall, Englewood Cliffs, 2002.


\end{document}
